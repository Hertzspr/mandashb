% Options for packages loaded elsewhere
\PassOptionsToPackage{unicode}{hyperref}
\PassOptionsToPackage{hyphens}{url}
%
\documentclass[
]{article}
\usepackage{amsmath,amssymb}
\usepackage{lmodern}
\usepackage{iftex}
\ifPDFTeX
  \usepackage[T1]{fontenc}
  \usepackage[utf8]{inputenc}
  \usepackage{textcomp} % provide euro and other symbols
\else % if luatex or xetex
  \usepackage{unicode-math}
  \defaultfontfeatures{Scale=MatchLowercase}
  \defaultfontfeatures[\rmfamily]{Ligatures=TeX,Scale=1}
\fi
% Use upquote if available, for straight quotes in verbatim environments
\IfFileExists{upquote.sty}{\usepackage{upquote}}{}
\IfFileExists{microtype.sty}{% use microtype if available
  \usepackage[]{microtype}
  \UseMicrotypeSet[protrusion]{basicmath} % disable protrusion for tt fonts
}{}
\makeatletter
\@ifundefined{KOMAClassName}{% if non-KOMA class
  \IfFileExists{parskip.sty}{%
    \usepackage{parskip}
  }{% else
    \setlength{\parindent}{0pt}
    \setlength{\parskip}{6pt plus 2pt minus 1pt}}
}{% if KOMA class
  \KOMAoptions{parskip=half}}
\makeatother
\usepackage{xcolor}
\usepackage[margin=1in]{geometry}
\usepackage{color}
\usepackage{fancyvrb}
\newcommand{\VerbBar}{|}
\newcommand{\VERB}{\Verb[commandchars=\\\{\}]}
\DefineVerbatimEnvironment{Highlighting}{Verbatim}{commandchars=\\\{\}}
% Add ',fontsize=\small' for more characters per line
\usepackage{framed}
\definecolor{shadecolor}{RGB}{248,248,248}
\newenvironment{Shaded}{\begin{snugshade}}{\end{snugshade}}
\newcommand{\AlertTok}[1]{\textcolor[rgb]{0.94,0.16,0.16}{#1}}
\newcommand{\AnnotationTok}[1]{\textcolor[rgb]{0.56,0.35,0.01}{\textbf{\textit{#1}}}}
\newcommand{\AttributeTok}[1]{\textcolor[rgb]{0.77,0.63,0.00}{#1}}
\newcommand{\BaseNTok}[1]{\textcolor[rgb]{0.00,0.00,0.81}{#1}}
\newcommand{\BuiltInTok}[1]{#1}
\newcommand{\CharTok}[1]{\textcolor[rgb]{0.31,0.60,0.02}{#1}}
\newcommand{\CommentTok}[1]{\textcolor[rgb]{0.56,0.35,0.01}{\textit{#1}}}
\newcommand{\CommentVarTok}[1]{\textcolor[rgb]{0.56,0.35,0.01}{\textbf{\textit{#1}}}}
\newcommand{\ConstantTok}[1]{\textcolor[rgb]{0.00,0.00,0.00}{#1}}
\newcommand{\ControlFlowTok}[1]{\textcolor[rgb]{0.13,0.29,0.53}{\textbf{#1}}}
\newcommand{\DataTypeTok}[1]{\textcolor[rgb]{0.13,0.29,0.53}{#1}}
\newcommand{\DecValTok}[1]{\textcolor[rgb]{0.00,0.00,0.81}{#1}}
\newcommand{\DocumentationTok}[1]{\textcolor[rgb]{0.56,0.35,0.01}{\textbf{\textit{#1}}}}
\newcommand{\ErrorTok}[1]{\textcolor[rgb]{0.64,0.00,0.00}{\textbf{#1}}}
\newcommand{\ExtensionTok}[1]{#1}
\newcommand{\FloatTok}[1]{\textcolor[rgb]{0.00,0.00,0.81}{#1}}
\newcommand{\FunctionTok}[1]{\textcolor[rgb]{0.00,0.00,0.00}{#1}}
\newcommand{\ImportTok}[1]{#1}
\newcommand{\InformationTok}[1]{\textcolor[rgb]{0.56,0.35,0.01}{\textbf{\textit{#1}}}}
\newcommand{\KeywordTok}[1]{\textcolor[rgb]{0.13,0.29,0.53}{\textbf{#1}}}
\newcommand{\NormalTok}[1]{#1}
\newcommand{\OperatorTok}[1]{\textcolor[rgb]{0.81,0.36,0.00}{\textbf{#1}}}
\newcommand{\OtherTok}[1]{\textcolor[rgb]{0.56,0.35,0.01}{#1}}
\newcommand{\PreprocessorTok}[1]{\textcolor[rgb]{0.56,0.35,0.01}{\textit{#1}}}
\newcommand{\RegionMarkerTok}[1]{#1}
\newcommand{\SpecialCharTok}[1]{\textcolor[rgb]{0.00,0.00,0.00}{#1}}
\newcommand{\SpecialStringTok}[1]{\textcolor[rgb]{0.31,0.60,0.02}{#1}}
\newcommand{\StringTok}[1]{\textcolor[rgb]{0.31,0.60,0.02}{#1}}
\newcommand{\VariableTok}[1]{\textcolor[rgb]{0.00,0.00,0.00}{#1}}
\newcommand{\VerbatimStringTok}[1]{\textcolor[rgb]{0.31,0.60,0.02}{#1}}
\newcommand{\WarningTok}[1]{\textcolor[rgb]{0.56,0.35,0.01}{\textbf{\textit{#1}}}}
\usepackage{longtable,booktabs,array}
\usepackage{calc} % for calculating minipage widths
% Correct order of tables after \paragraph or \subparagraph
\usepackage{etoolbox}
\makeatletter
\patchcmd\longtable{\par}{\if@noskipsec\mbox{}\fi\par}{}{}
\makeatother
% Allow footnotes in longtable head/foot
\IfFileExists{footnotehyper.sty}{\usepackage{footnotehyper}}{\usepackage{footnote}}
\makesavenoteenv{longtable}
\usepackage{graphicx}
\makeatletter
\def\maxwidth{\ifdim\Gin@nat@width>\linewidth\linewidth\else\Gin@nat@width\fi}
\def\maxheight{\ifdim\Gin@nat@height>\textheight\textheight\else\Gin@nat@height\fi}
\makeatother
% Scale images if necessary, so that they will not overflow the page
% margins by default, and it is still possible to overwrite the defaults
% using explicit options in \includegraphics[width, height, ...]{}
\setkeys{Gin}{width=\maxwidth,height=\maxheight,keepaspectratio}
% Set default figure placement to htbp
\makeatletter
\def\fps@figure{htbp}
\makeatother
\setlength{\emergencystretch}{3em} % prevent overfull lines
\providecommand{\tightlist}{%
  \setlength{\itemsep}{0pt}\setlength{\parskip}{0pt}}
\setcounter{secnumdepth}{-\maxdimen} % remove section numbering
\ifLuaTeX
  \usepackage{selnolig}  % disable illegal ligatures
\fi
\IfFileExists{bookmark.sty}{\usepackage{bookmark}}{\usepackage{hyperref}}
\IfFileExists{xurl.sty}{\usepackage{xurl}}{} % add URL line breaks if available
\urlstyle{same} % disable monospaced font for URLs
\hypersetup{
  pdftitle={Siswa Bahasa Indonesia Noni},
  pdfauthor={S.Y. Husada},
  hidelinks,
  pdfcreator={LaTeX via pandoc}}

\title{Siswa Bahasa Indonesia Noni}
\author{S.Y. Husada}
\date{2022-08-18}

\begin{document}
\maketitle

\hypertarget{overview}{%
\section{Overview}\label{overview}}

\hypertarget{objective}{%
\subsection{Objective}\label{objective}}

This is an exploratory analysis on three classes of students scores in
Indonesian language subject. These classes were under the tutelage of A.
N. in MAN 3 Lebak and I hope I can discover something of value from this
analysis.

\hypertarget{data}{%
\subsection{Data}\label{data}}

The data consists of prepared tables that I had processed in excel, they
are:

\begin{enumerate}
\def\labelenumi{\arabic{enumi}.}
\item
  \texttt{harian} table, that presents 4 sets of minor tests result that
  each consists of \texttt{proyek}, \texttt{portofolio}, and
  \texttt{praktek};
\item
  \texttt{phbo\ pat} table that presents 2 sets of major (half and last
  semester) tests result.
\end{enumerate}

\hypertarget{preparation}{%
\section{Preparation}\label{preparation}}

\hypertarget{library}{%
\subsection{Library}\label{library}}

\begin{Shaded}
\begin{Highlighting}[]
\FunctionTok{library}\NormalTok{(readxl)}
\FunctionTok{library}\NormalTok{(tidyverse)}
\FunctionTok{library}\NormalTok{(tidymodels)}
\FunctionTok{library}\NormalTok{(encryptr)}
\FunctionTok{library}\NormalTok{(janitor)}
\FunctionTok{library}\NormalTok{(skimr)}
\FunctionTok{library}\NormalTok{(plotly)}
\FunctionTok{library}\NormalTok{(glue)}
\FunctionTok{library}\NormalTok{(GGally)}
\FunctionTok{library}\NormalTok{(tidytext)}
\FunctionTok{library}\NormalTok{(FactoMineR)}
\FunctionTok{library}\NormalTok{(ggiraphExtra)}
\FunctionTok{library}\NormalTok{(factoextra)}
\end{Highlighting}
\end{Shaded}

\hypertarget{import}{%
\subsection{Import}\label{import}}

\begin{Shaded}
\begin{Highlighting}[]
\NormalTok{harian }\OtherTok{\textless{}{-}} \FunctionTok{read\_excel}\NormalTok{(}\StringTok{"edit\_bi\_kelas\_noni.xlsx"}\NormalTok{, }
    \AttributeTok{sheet =} \StringTok{"harian"}\NormalTok{)}
\NormalTok{semesteran }\OtherTok{\textless{}{-}} \FunctionTok{read\_excel}\NormalTok{(}\StringTok{"edit\_bi\_kelas\_noni.xlsx"}\NormalTok{, }
    \AttributeTok{sheet =} \StringTok{"phbo pat"}\NormalTok{)}
\end{Highlighting}
\end{Shaded}

Clean the column names.

\begin{Shaded}
\begin{Highlighting}[]
\NormalTok{harian }\OtherTok{\textless{}{-}} \FunctionTok{clean\_names}\NormalTok{(harian)}
\NormalTok{semesteran }\OtherTok{\textless{}{-}} \FunctionTok{clean\_names}\NormalTok{(semesteran)}
\end{Highlighting}
\end{Shaded}

\hypertarget{encryption}{%
\subsection{Encryption}\label{encryption}}

Encrypt the name of students to protect privacy.

\hypertarget{generate-keys}{%
\subsubsection{Generate Keys}\label{generate-keys}}

\begin{Shaded}
\begin{Highlighting}[]
\CommentTok{\#genkeys()}
\end{Highlighting}
\end{Shaded}

\hypertarget{encrypt-names}{%
\subsubsection{Encrypt Names}\label{encrypt-names}}

Create codes for names.

\begin{Shaded}
\begin{Highlighting}[]
\NormalTok{names\_encryption }\OtherTok{\textless{}{-}} 
\NormalTok{  harian }\SpecialCharTok{\%\textgreater{}\%} 
  \FunctionTok{select}\NormalTok{(nama\_siswa) }\SpecialCharTok{\%\textgreater{}\%}
  \FunctionTok{unique}\NormalTok{() }\SpecialCharTok{\%\textgreater{}\%} 
  \FunctionTok{mutate}\NormalTok{(}
    \AttributeTok{name\_code =}\NormalTok{ nama\_siswa}
\NormalTok{  ) }\SpecialCharTok{\%\textgreater{}\%} 
  \FunctionTok{encrypt}\NormalTok{(name\_code)}
\end{Highlighting}
\end{Shaded}

Replace real names in \texttt{nama\_siswa} with the codes.

\begin{Shaded}
\begin{Highlighting}[]
\NormalTok{harian\_encrypted }\OtherTok{\textless{}{-}} 
\NormalTok{  harian }\SpecialCharTok{\%\textgreater{}\%} 
  \FunctionTok{left\_join}\NormalTok{(names\_encryption, }
            \AttributeTok{by =} \FunctionTok{c}\NormalTok{(}\StringTok{"nama\_siswa"} \OtherTok{=} \StringTok{"nama\_siswa"}\NormalTok{),}
            \AttributeTok{keep =}\NormalTok{ F) }\SpecialCharTok{\%\textgreater{}\%} 
  \FunctionTok{select}\NormalTok{(}\SpecialCharTok{{-}}\NormalTok{nama\_siswa) }\SpecialCharTok{\%\textgreater{}\%} 
  \FunctionTok{select}\NormalTok{(name\_code, }\FunctionTok{everything}\NormalTok{())}


\NormalTok{harian\_encrypted }\SpecialCharTok{\%\textgreater{}\%} \FunctionTok{head}\NormalTok{()}
\end{Highlighting}
\end{Shaded}

\begin{verbatim}
## # A tibble: 6 x 8
##   name_code                  kelas penil~1 materi nilai~2 proyek praktek porto~3
##   <chr>                      <chr> <chr>   <chr>    <dbl>  <dbl>   <dbl>   <dbl>
## 1 05d012f707a50ab936f3d52b3~ XI I~ Harian~ Propo~      78     78      78      78
## 2 4128d7d128e73f2b520fd464b~ XI I~ Harian~ Propo~      78     78      78      78
## 3 08d87310b7c81d621f0299cdc~ XI I~ Harian~ Propo~      78     78      78      78
## 4 adddc0a893fe566b9671f7e8f~ XI I~ Harian~ Propo~      78     78      78      78
## 5 a8921a405e52d4e732485cea7~ XI I~ Harian~ Propo~      81     80      83      80
## 6 0d4fa26847745631c4df3be6b~ XI I~ Harian~ Propo~      78     78      78      78
## # ... with abbreviated variable names 1: penilaian, 2: nilai_ph, 3: portofolio
\end{verbatim}

Compare total unique \texttt{nama\_siswa} and unique \texttt{name\_code}
to check.

\begin{Shaded}
\begin{Highlighting}[]
\FunctionTok{n\_distinct}\NormalTok{(harian}\SpecialCharTok{$}\NormalTok{nama\_siswa)}
\end{Highlighting}
\end{Shaded}

\begin{verbatim}
## [1] 82
\end{verbatim}

\begin{Shaded}
\begin{Highlighting}[]
\FunctionTok{n\_distinct}\NormalTok{(harian\_encrypted}\SpecialCharTok{$}\NormalTok{name\_code)}
\end{Highlighting}
\end{Shaded}

\begin{verbatim}
## [1] 82
\end{verbatim}

Encrypt \texttt{semesteran} table as well.

\begin{Shaded}
\begin{Highlighting}[]
\NormalTok{semesteran\_encrypted }\OtherTok{\textless{}{-}} 
\NormalTok{  semesteran }\SpecialCharTok{\%\textgreater{}\%} 
  \FunctionTok{left\_join}\NormalTok{(names\_encryption, }
            \AttributeTok{by =} \FunctionTok{c}\NormalTok{(}\StringTok{"nama\_siswa"} \OtherTok{=} \StringTok{"nama\_siswa"}\NormalTok{),}
            \AttributeTok{keep =}\NormalTok{ F) }\SpecialCharTok{\%\textgreater{}\%} 
  \FunctionTok{select}\NormalTok{(}\SpecialCharTok{{-}}\NormalTok{nama\_siswa) }\SpecialCharTok{\%\textgreater{}\%} 
  \FunctionTok{select}\NormalTok{(name\_code, }\FunctionTok{everything}\NormalTok{())}


\NormalTok{semesteran\_encrypted }\SpecialCharTok{\%\textgreater{}\%} \FunctionTok{head}\NormalTok{()}
\end{Highlighting}
\end{Shaded}

\begin{verbatim}
## # A tibble: 6 x 4
##   name_code                                                  kelas penil~1 nilai
##   <chr>                                                      <chr> <chr>   <dbl>
## 1 05d012f707a50ab936f3d52b33582adccf7e015fe830b1c9d2624d4e3~ XI I~ PHBO       78
## 2 4128d7d128e73f2b520fd464b9bcef5294745c625ef3b103566a51270~ XI I~ PHBO       78
## 3 08d87310b7c81d621f0299cdcc647fb7a4f67a9f3c6984be4841cb516~ XI I~ PHBO       78
## 4 adddc0a893fe566b9671f7e8f0de309c10f40b4294d5a55dd32f6a911~ XI I~ PHBO       78
## 5 a8921a405e52d4e732485cea7f36829f426faa028fb0e1d257eb1507d~ XI I~ PHBO       98
## 6 0d4fa26847745631c4df3be6b4f3fc99f9893d8dd0d30b526da99500c~ XI I~ PHBO       78
## # ... with abbreviated variable name 1: penilaian
\end{verbatim}

\begin{Shaded}
\begin{Highlighting}[]
\FunctionTok{n\_distinct}\NormalTok{(semesteran}\SpecialCharTok{$}\NormalTok{nama\_siswa)}
\end{Highlighting}
\end{Shaded}

\begin{verbatim}
## [1] 82
\end{verbatim}

\begin{Shaded}
\begin{Highlighting}[]
\FunctionTok{n\_distinct}\NormalTok{(semesteran\_encrypted}\SpecialCharTok{$}\NormalTok{name\_code)}
\end{Highlighting}
\end{Shaded}

\begin{verbatim}
## [1] 82
\end{verbatim}

I had processed it in excel, but let's check missing data again.

\begin{Shaded}
\begin{Highlighting}[]
\FunctionTok{anyNA}\NormalTok{(harian\_encrypted)}
\end{Highlighting}
\end{Shaded}

\begin{verbatim}
## [1] FALSE
\end{verbatim}

\begin{Shaded}
\begin{Highlighting}[]
\FunctionTok{anyNA}\NormalTok{(semesteran\_encrypted)}
\end{Highlighting}
\end{Shaded}

\begin{verbatim}
## [1] FALSE
\end{verbatim}

\hypertarget{shuffle}{%
\subsubsection{Shuffle}\label{shuffle}}

Shuffle the row so the real names can't be identified using the order of
the original rows.

\begin{Shaded}
\begin{Highlighting}[]
\NormalTok{harian\_encrypted }\OtherTok{\textless{}{-}}\NormalTok{ harian\_encrypted }\SpecialCharTok{\%\textgreater{}\%} \FunctionTok{slice\_sample}\NormalTok{(}\AttributeTok{prop =} \DecValTok{1}\NormalTok{)}
\NormalTok{semesteran\_encrypted }\OtherTok{\textless{}{-}}\NormalTok{ semesteran\_encrypted }\SpecialCharTok{\%\textgreater{}\%} \FunctionTok{slice\_sample}\NormalTok{(}\AttributeTok{prop =} \DecValTok{1}\NormalTok{)}
\end{Highlighting}
\end{Shaded}

\hypertarget{analysis}{%
\section{Analysis}\label{analysis}}

\hypertarget{exploratory}{%
\subsection{Exploratory}\label{exploratory}}

\begin{Shaded}
\begin{Highlighting}[]
\FunctionTok{glimpse}\NormalTok{(harian\_encrypted)}
\end{Highlighting}
\end{Shaded}

\begin{verbatim}
## Rows: 328
## Columns: 8
## $ name_code  <chr> "00eef26f828f1aaae411f94c8e475927548447fb48160f5b928fa3e9fd~
## $ kelas      <chr> "XI IPS 1", "XI IPS 1", "XI IPS 2", "XI MIPA", "XI IPS 1", ~
## $ penilaian  <chr> "Harian 4", "Harian 2", "Harian 3", "Harian 3", "Harian 3",~
## $ materi     <chr> "Drama", "Karya Ilmiah", "Resensi", "Resensi", "Resensi", "~
## $ nilai_ph   <dbl> 85.00000, 78.00000, 83.33333, 85.00000, 79.33333, 78.00000,~
## $ proyek     <dbl> 85, 78, 85, 85, 78, 78, 78, 85, 78, 78, 78, 80, 80, 78, 85,~
## $ praktek    <dbl> 85, 78, 85, 85, 78, 78, 78, 85, 78, 78, 78, 90, 85, 78, 85,~
## $ portofolio <dbl> 85, 78, 80, 85, 82, 78, 78, 78, 78, 78, 78, 80, 80, 78, 80,~
\end{verbatim}

\begin{Shaded}
\begin{Highlighting}[]
\FunctionTok{glimpse}\NormalTok{(semesteran\_encrypted)}
\end{Highlighting}
\end{Shaded}

\begin{verbatim}
## Rows: 164
## Columns: 4
## $ name_code <chr> "72da0cd086fa188e3746b03b1b4a3b7bb0619768bbf98d3230e2e20ebec~
## $ kelas     <chr> "XI MIPA", "XI IPS 2", "XI MIPA", "XI IPS 2", "XI IPS 2", "X~
## $ penilaian <chr> "PAT", "PAT", "PHBO", "PAT", "PHBO", "PHBO", "PAT", "PHBO", ~
## $ nilai     <dbl> 83, 78, 92, 82, 78, 78, 84, 85, 88, 86, 78, 88, 78, 80, 92, ~
\end{verbatim}

We can join the tables with name\_code.

\begin{Shaded}
\begin{Highlighting}[]
\NormalTok{data }\OtherTok{\textless{}{-}}
\NormalTok{  harian\_encrypted }\SpecialCharTok{\%\textgreater{}\%} \FunctionTok{left\_join}\NormalTok{(semesteran\_encrypted[,}\FunctionTok{c}\NormalTok{(}\DecValTok{1}\NormalTok{, }\DecValTok{3}\NormalTok{, }\DecValTok{4}\NormalTok{)], }
                                 \AttributeTok{by =} \StringTok{"name\_code"}\NormalTok{,}
                                 \AttributeTok{suffix =} \FunctionTok{c}\NormalTok{(}\StringTok{"\_h"}\NormalTok{, }\StringTok{"\_s"}\NormalTok{))}
\end{Highlighting}
\end{Shaded}

\begin{Shaded}
\begin{Highlighting}[]
\FunctionTok{glimpse}\NormalTok{(data)}
\end{Highlighting}
\end{Shaded}

\begin{verbatim}
## Rows: 656
## Columns: 10
## $ name_code   <chr> "00eef26f828f1aaae411f94c8e475927548447fb48160f5b928fa3e9f~
## $ kelas       <chr> "XI IPS 1", "XI IPS 1", "XI IPS 1", "XI IPS 1", "XI IPS 2"~
## $ penilaian_h <chr> "Harian 4", "Harian 4", "Harian 2", "Harian 2", "Harian 3"~
## $ materi      <chr> "Drama", "Drama", "Karya Ilmiah", "Karya Ilmiah", "Resensi~
## $ nilai_ph    <dbl> 85.00000, 85.00000, 78.00000, 78.00000, 83.33333, 83.33333~
## $ proyek      <dbl> 85, 85, 78, 78, 85, 85, 85, 85, 78, 78, 78, 78, 78, 78, 85~
## $ praktek     <dbl> 85, 85, 78, 78, 85, 85, 85, 85, 78, 78, 78, 78, 78, 78, 85~
## $ portofolio  <dbl> 85, 85, 78, 78, 80, 80, 85, 85, 82, 82, 78, 78, 78, 78, 78~
## $ penilaian_s <chr> "PAT", "PHBO", "PHBO", "PAT", "PHBO", "PAT", "PHBO", "PAT"~
## $ nilai       <dbl> 91, 93, 78, 86, 98, 89, 95, 83, 87, 88, 86, 83, 98, 81, 85~
\end{verbatim}

Convert all the chr columns except name\_code, to factor.

\begin{Shaded}
\begin{Highlighting}[]
\NormalTok{data }\OtherTok{\textless{}{-}}\NormalTok{ data }\SpecialCharTok{\%\textgreater{}\%} 
  \FunctionTok{mutate}\NormalTok{(}\FunctionTok{across}\NormalTok{(}\AttributeTok{.cols =} \FunctionTok{c}\NormalTok{(}\FunctionTok{where}\NormalTok{(is.character), }\SpecialCharTok{{-}}\NormalTok{name\_code),}\AttributeTok{.fns =}\NormalTok{ as.factor))}
\end{Highlighting}
\end{Shaded}

\begin{Shaded}
\begin{Highlighting}[]
\FunctionTok{glimpse}\NormalTok{(data)}
\end{Highlighting}
\end{Shaded}

\begin{verbatim}
## Rows: 656
## Columns: 10
## $ name_code   <chr> "00eef26f828f1aaae411f94c8e475927548447fb48160f5b928fa3e9f~
## $ kelas       <fct> XI IPS 1, XI IPS 1, XI IPS 1, XI IPS 1, XI IPS 2, XI IPS 2~
## $ penilaian_h <fct> Harian 4, Harian 4, Harian 2, Harian 2, Harian 3, Harian 3~
## $ materi      <fct> Drama, Drama, Karya Ilmiah, Karya Ilmiah, Resensi, Resensi~
## $ nilai_ph    <dbl> 85.00000, 85.00000, 78.00000, 78.00000, 83.33333, 83.33333~
## $ proyek      <dbl> 85, 85, 78, 78, 85, 85, 85, 85, 78, 78, 78, 78, 78, 78, 85~
## $ praktek     <dbl> 85, 85, 78, 78, 85, 85, 85, 85, 78, 78, 78, 78, 78, 78, 85~
## $ portofolio  <dbl> 85, 85, 78, 78, 80, 80, 85, 85, 82, 82, 78, 78, 78, 78, 78~
## $ penilaian_s <fct> PAT, PHBO, PHBO, PAT, PHBO, PAT, PHBO, PAT, PAT, PHBO, PHB~
## $ nilai       <dbl> 91, 93, 78, 86, 98, 89, 95, 83, 87, 88, 86, 83, 98, 81, 85~
\end{verbatim}

\begin{Shaded}
\begin{Highlighting}[]
\FunctionTok{skim}\NormalTok{(}\AttributeTok{data =}\NormalTok{ data)}
\end{Highlighting}
\end{Shaded}

\begin{longtable}[]{@{}ll@{}}
\caption{Data summary}\tabularnewline
\toprule()
\endhead
Name & data \\
Number of rows & 656 \\
Number of columns & 10 \\
\_\_\_\_\_\_\_\_\_\_\_\_\_\_\_\_\_\_\_\_\_\_\_ & \\
Column type frequency: & \\
character & 1 \\
factor & 4 \\
numeric & 5 \\
\_\_\_\_\_\_\_\_\_\_\_\_\_\_\_\_\_\_\_\_\_\_\_\_ & \\
Group variables & None \\
\bottomrule()
\end{longtable}

\textbf{Variable type: character}

\begin{longtable}[]{@{}
  >{\raggedright\arraybackslash}p{(\columnwidth - 14\tabcolsep) * \real{0.1944}}
  >{\raggedleft\arraybackslash}p{(\columnwidth - 14\tabcolsep) * \real{0.1389}}
  >{\raggedleft\arraybackslash}p{(\columnwidth - 14\tabcolsep) * \real{0.1944}}
  >{\raggedleft\arraybackslash}p{(\columnwidth - 14\tabcolsep) * \real{0.0556}}
  >{\raggedleft\arraybackslash}p{(\columnwidth - 14\tabcolsep) * \real{0.0556}}
  >{\raggedleft\arraybackslash}p{(\columnwidth - 14\tabcolsep) * \real{0.0833}}
  >{\raggedleft\arraybackslash}p{(\columnwidth - 14\tabcolsep) * \real{0.1250}}
  >{\raggedleft\arraybackslash}p{(\columnwidth - 14\tabcolsep) * \real{0.1528}}@{}}
\toprule()
\begin{minipage}[b]{\linewidth}\raggedright
skim\_variable
\end{minipage} & \begin{minipage}[b]{\linewidth}\raggedleft
n\_missing
\end{minipage} & \begin{minipage}[b]{\linewidth}\raggedleft
complete\_rate
\end{minipage} & \begin{minipage}[b]{\linewidth}\raggedleft
min
\end{minipage} & \begin{minipage}[b]{\linewidth}\raggedleft
max
\end{minipage} & \begin{minipage}[b]{\linewidth}\raggedleft
empty
\end{minipage} & \begin{minipage}[b]{\linewidth}\raggedleft
n\_unique
\end{minipage} & \begin{minipage}[b]{\linewidth}\raggedleft
whitespace
\end{minipage} \\
\midrule()
\endhead
name\_code & 0 & 1 & 512 & 512 & 0 & 82 & 0 \\
\bottomrule()
\end{longtable}

\textbf{Variable type: factor}

\begin{longtable}[]{@{}
  >{\raggedright\arraybackslash}p{(\columnwidth - 10\tabcolsep) * \real{0.1489}}
  >{\raggedleft\arraybackslash}p{(\columnwidth - 10\tabcolsep) * \real{0.1064}}
  >{\raggedleft\arraybackslash}p{(\columnwidth - 10\tabcolsep) * \real{0.1489}}
  >{\raggedright\arraybackslash}p{(\columnwidth - 10\tabcolsep) * \real{0.0851}}
  >{\raggedleft\arraybackslash}p{(\columnwidth - 10\tabcolsep) * \real{0.0957}}
  >{\raggedright\arraybackslash}p{(\columnwidth - 10\tabcolsep) * \real{0.4149}}@{}}
\toprule()
\begin{minipage}[b]{\linewidth}\raggedright
skim\_variable
\end{minipage} & \begin{minipage}[b]{\linewidth}\raggedleft
n\_missing
\end{minipage} & \begin{minipage}[b]{\linewidth}\raggedleft
complete\_rate
\end{minipage} & \begin{minipage}[b]{\linewidth}\raggedright
ordered
\end{minipage} & \begin{minipage}[b]{\linewidth}\raggedleft
n\_unique
\end{minipage} & \begin{minipage}[b]{\linewidth}\raggedright
top\_counts
\end{minipage} \\
\midrule()
\endhead
kelas & 0 & 1 & FALSE & 3 & XI : 256, XI : 208, XI : 192 \\
penilaian\_h & 0 & 1 & FALSE & 4 & Har: 164, Har: 164, Har: 164, Har:
164 \\
materi & 0 & 1 & FALSE & 4 & Dra: 164, Kar: 164, Pro: 164, Res: 164 \\
penilaian\_s & 0 & 1 & FALSE & 2 & PAT: 328, PHB: 328 \\
\bottomrule()
\end{longtable}

\textbf{Variable type: numeric}

\begin{longtable}[]{@{}
  >{\raggedright\arraybackslash}p{(\columnwidth - 20\tabcolsep) * \real{0.1228}}
  >{\raggedleft\arraybackslash}p{(\columnwidth - 20\tabcolsep) * \real{0.0877}}
  >{\raggedleft\arraybackslash}p{(\columnwidth - 20\tabcolsep) * \real{0.1228}}
  >{\raggedleft\arraybackslash}p{(\columnwidth - 20\tabcolsep) * \real{0.0526}}
  >{\raggedleft\arraybackslash}p{(\columnwidth - 20\tabcolsep) * \real{0.0439}}
  >{\raggedleft\arraybackslash}p{(\columnwidth - 20\tabcolsep) * \real{0.0263}}
  >{\raggedleft\arraybackslash}p{(\columnwidth - 20\tabcolsep) * \real{0.0351}}
  >{\raggedleft\arraybackslash}p{(\columnwidth - 20\tabcolsep) * \real{0.0351}}
  >{\raggedleft\arraybackslash}p{(\columnwidth - 20\tabcolsep) * \real{0.0526}}
  >{\raggedleft\arraybackslash}p{(\columnwidth - 20\tabcolsep) * \real{0.0614}}
  >{\raggedright\arraybackslash}p{(\columnwidth - 20\tabcolsep) * \real{0.3596}}@{}}
\toprule()
\begin{minipage}[b]{\linewidth}\raggedright
skim\_variable
\end{minipage} & \begin{minipage}[b]{\linewidth}\raggedleft
n\_missing
\end{minipage} & \begin{minipage}[b]{\linewidth}\raggedleft
complete\_rate
\end{minipage} & \begin{minipage}[b]{\linewidth}\raggedleft
mean
\end{minipage} & \begin{minipage}[b]{\linewidth}\raggedleft
sd
\end{minipage} & \begin{minipage}[b]{\linewidth}\raggedleft
p0
\end{minipage} & \begin{minipage}[b]{\linewidth}\raggedleft
p25
\end{minipage} & \begin{minipage}[b]{\linewidth}\raggedleft
p50
\end{minipage} & \begin{minipage}[b]{\linewidth}\raggedleft
p75
\end{minipage} & \begin{minipage}[b]{\linewidth}\raggedleft
p100
\end{minipage} & \begin{minipage}[b]{\linewidth}\raggedright
hist
\end{minipage} \\
\midrule()
\endhead
nilai\_ph & 0 & 1 & 79.92 & 3.15 & 78 & 78 & 78 & 81.67 & 92.67 &
▇▁▂▁▁ \\
proyek & 0 & 1 & 80.19 & 4.01 & 78 & 78 & 78 & 80.00 & 100.00 & ▇▂▁▁▁ \\
praktek & 0 & 1 & 79.70 & 3.03 & 78 & 78 & 78 & 80.00 & 90.00 & ▇▁▂▁▁ \\
portofolio & 0 & 1 & 79.88 & 3.83 & 78 & 78 & 78 & 80.00 & 100.00 &
▇▁▁▁▁ \\
nilai & 0 & 1 & 84.57 & 5.33 & 78 & 78 & 85 & 88.00 & 98.00 & ▇▇▅▂▁ \\
\bottomrule()
\end{longtable}

\begin{Shaded}
\begin{Highlighting}[]
\FunctionTok{summary}\NormalTok{(data)}
\end{Highlighting}
\end{Shaded}

\begin{verbatim}
##   name_code              kelas       penilaian_h                materi   
##  Length:656         XI IPS 1:208   Harian 1:164   Drama            :164  
##  Class :character   XI IPS 2:192   Harian 2:164   Karya Ilmiah     :164  
##  Mode  :character   XI MIPA :256   Harian 3:164   Proposal Kegiatan:164  
##                                    Harian 4:164   Resensi          :164  
##                                                                          
##                                                                          
##     nilai_ph         proyek          praktek       portofolio     penilaian_s
##  Min.   :78.00   Min.   : 78.00   Min.   :78.0   Min.   : 78.00   PAT :328   
##  1st Qu.:78.00   1st Qu.: 78.00   1st Qu.:78.0   1st Qu.: 78.00   PHBO:328   
##  Median :78.00   Median : 78.00   Median :78.0   Median : 78.00              
##  Mean   :79.92   Mean   : 80.19   Mean   :79.7   Mean   : 79.88              
##  3rd Qu.:81.67   3rd Qu.: 80.00   3rd Qu.:80.0   3rd Qu.: 80.00              
##  Max.   :92.67   Max.   :100.00   Max.   :90.0   Max.   :100.00              
##      nilai      
##  Min.   :78.00  
##  1st Qu.:78.00  
##  Median :85.00  
##  Mean   :84.57  
##  3rd Qu.:88.00  
##  Max.   :98.00
\end{verbatim}

\hypertarget{nilai-rata-rata-harian-per-kelas}{%
\subsubsection{Nilai Rata-rata Harian per
Kelas}\label{nilai-rata-rata-harian-per-kelas}}

\begin{Shaded}
\begin{Highlighting}[]
\NormalTok{data }\OtherTok{\textless{}{-}}\NormalTok{ data }\SpecialCharTok{\%\textgreater{}\%} 
  \FunctionTok{pivot\_longer}\NormalTok{(}\AttributeTok{cols =} \DecValTok{6}\SpecialCharTok{:}\DecValTok{8}\NormalTok{,}
               \AttributeTok{names\_to =} \StringTok{"ppp"}\NormalTok{,}
               \AttributeTok{values\_to =} \StringTok{"nilai\_ppp"}\NormalTok{) }
\end{Highlighting}
\end{Shaded}

\begin{Shaded}
\begin{Highlighting}[]
\NormalTok{data\_mean\_ph }\OtherTok{\textless{}{-}}
\NormalTok{  data }\SpecialCharTok{\%\textgreater{}\%} 
  \FunctionTok{group\_by}\NormalTok{(kelas, penilaian\_h, materi, ppp) }\SpecialCharTok{\%\textgreater{}\%} 
  \FunctionTok{summarise}\NormalTok{(}\AttributeTok{rata2\_ph =} \FunctionTok{round}\NormalTok{(}\FunctionTok{mean}\NormalTok{(nilai\_ph),}\AttributeTok{digits =} \DecValTok{2}\NormalTok{)) }\SpecialCharTok{\%\textgreater{}\%} 
  \FunctionTok{mutate}\NormalTok{(}\AttributeTok{avg =} \FunctionTok{glue}\NormalTok{(}\StringTok{"Nilai Rata{-}rata \{rata2\_ph\}"}\NormalTok{))}
\end{Highlighting}
\end{Shaded}

\begin{Shaded}
\begin{Highlighting}[]
\NormalTok{acols }\OtherTok{\textless{}{-}} \FunctionTok{c}\NormalTok{(}\StringTok{"Harian 1"} \OtherTok{=} \StringTok{"\#C6E0FF"}\NormalTok{, }
           \StringTok{"Harian 2"} \OtherTok{=} \StringTok{"\#579A9E"}\NormalTok{,}
           \StringTok{"Harian 3"} \OtherTok{=} \StringTok{"\#3292C3"}\NormalTok{,}
           \StringTok{"Harian 4"} \OtherTok{=} \StringTok{"\#BCAB79"}\NormalTok{)}

\NormalTok{rata.plot }\OtherTok{\textless{}{-}} \FunctionTok{ggplot}\NormalTok{(data\_mean\_ph, }\FunctionTok{aes}\NormalTok{(}\AttributeTok{y =}\NormalTok{ kelas, }\AttributeTok{x =}\NormalTok{ rata2\_ph, }\AttributeTok{fill =}\NormalTok{ penilaian\_h))}\SpecialCharTok{+}
  \FunctionTok{geom\_col}\NormalTok{(}\FunctionTok{aes}\NormalTok{(}\AttributeTok{text =}\NormalTok{ avg))}\SpecialCharTok{+}
  \FunctionTok{facet\_wrap}\NormalTok{(}\SpecialCharTok{\textasciitilde{}}\NormalTok{penilaian\_h)}\SpecialCharTok{+}
  \FunctionTok{theme\_minimal}\NormalTok{()}\SpecialCharTok{+}
  \FunctionTok{theme}\NormalTok{(}\AttributeTok{legend.position =} \StringTok{"none"}\NormalTok{,}
        \AttributeTok{axis.title.y =} \FunctionTok{element\_blank}\NormalTok{())}\SpecialCharTok{+}
  \FunctionTok{labs}\NormalTok{(}\AttributeTok{title =} \StringTok{"Perbandingan Rata{-}rata Nilai Harian Pelajaran Bahasa"}\NormalTok{,}
       \AttributeTok{subtitle =} \StringTok{"2020/2021 Semester Genap"}\NormalTok{,}
       \AttributeTok{x =} \StringTok{"Nilai"}\NormalTok{)}\SpecialCharTok{+}
  \FunctionTok{scale\_fill\_manual}\NormalTok{(}
    \AttributeTok{values =}\NormalTok{ acols      }
\NormalTok{  )}\SpecialCharTok{+}
  \FunctionTok{xlim}\NormalTok{(}\DecValTok{0}\NormalTok{, }\DecValTok{100}\NormalTok{)}

\NormalTok{rata.plot}
\end{Highlighting}
\end{Shaded}

\includegraphics{unsupervised-learning_files/figure-latex/unnamed-chunk-21-1.pdf}

\begin{Shaded}
\begin{Highlighting}[]
\FunctionTok{ggplotly}\NormalTok{(rata.plot, }\AttributeTok{tooltip =} \StringTok{"text"}\NormalTok{)}
\end{Highlighting}
\end{Shaded}

\includegraphics{unsupervised-learning_files/figure-latex/unnamed-chunk-22-1.pdf}
\#\#\# Nilai Ranah Keterampilan

\begin{Shaded}
\begin{Highlighting}[]
\NormalTok{data }\OtherTok{\textless{}{-}}\NormalTok{ data }\SpecialCharTok{\%\textgreater{}\%} 
  \FunctionTok{mutate}\NormalTok{( }\CommentTok{\# repair ppp content}
    \AttributeTok{ppp =} \FunctionTok{str\_replace}\NormalTok{(ppp, }\StringTok{"p"}\NormalTok{, }\StringTok{"P"}\NormalTok{ ) }
\NormalTok{  ) }\SpecialCharTok{\%\textgreater{}\%} 
  \FunctionTok{mutate}\NormalTok{( }\CommentTok{\# limit digits}
    \AttributeTok{nilai\_ph =} \FunctionTok{round}\NormalTok{(nilai\_ph, }\AttributeTok{digits =} \DecValTok{2}\NormalTok{)}
\NormalTok{  ) }\SpecialCharTok{\%\textgreater{}\%} 
    \FunctionTok{mutate}\NormalTok{( }\CommentTok{\# creating tooltip}
    \AttributeTok{exp =} \FunctionTok{glue}\NormalTok{(}
      \StringTok{"\{kelas\} }
\StringTok{      \{penilaian\_h\}: \{nilai\_ph\}}
\StringTok{      \{ppp\}: \{nilai\_ppp\}"}
\NormalTok{    )}
\NormalTok{  )}
\end{Highlighting}
\end{Shaded}

\begin{Shaded}
\begin{Highlighting}[]
\NormalTok{fcols }\OtherTok{\textless{}{-}} \FunctionTok{c}\NormalTok{(}\StringTok{"XI IPS 1"} \OtherTok{=} \StringTok{"\#BEEF9E"}\NormalTok{, }
           \StringTok{"XI IPS 2"} \OtherTok{=} \StringTok{"\#3292C3"}\NormalTok{,}
           \StringTok{"XI MIPA"} \OtherTok{=} \StringTok{"\#BCAB79"}\NormalTok{)}

\NormalTok{ppp.plot }\OtherTok{\textless{}{-}} \FunctionTok{ggplot}\NormalTok{(data, }\FunctionTok{aes}\NormalTok{(}\AttributeTok{y =}\NormalTok{ materi, }
                 \AttributeTok{x =}\NormalTok{ nilai\_ppp, }
                 \AttributeTok{fill =}\NormalTok{ kelas))}\SpecialCharTok{+}
  \FunctionTok{geom\_col}\NormalTok{(}\AttributeTok{position =} \FunctionTok{position\_dodge}\NormalTok{(}\AttributeTok{width =} \FloatTok{0.95}\NormalTok{),}
           \FunctionTok{aes}\NormalTok{(}\AttributeTok{text =}\NormalTok{ exp))}\SpecialCharTok{+}
  \FunctionTok{facet\_wrap}\NormalTok{(}\SpecialCharTok{\textasciitilde{}}\NormalTok{ppp)}\SpecialCharTok{+}
  \FunctionTok{theme\_minimal}\NormalTok{()}\SpecialCharTok{+}
  \FunctionTok{theme}\NormalTok{(}\AttributeTok{legend.position =} \StringTok{"none"}\NormalTok{,}
        \AttributeTok{axis.title.y =} \FunctionTok{element\_blank}\NormalTok{())}\SpecialCharTok{+}
  \FunctionTok{labs}\NormalTok{(}\AttributeTok{title =} \StringTok{"Perbandingan Rincian Nilai Tugas Pelajaran Bahasa"}\NormalTok{,}
       \AttributeTok{subtitle =} \StringTok{"2020/2021 Semester Genap"}\NormalTok{,}
       \AttributeTok{x =} \StringTok{"Nilai"}\NormalTok{)}\SpecialCharTok{+}
  \FunctionTok{scale\_fill\_manual}\NormalTok{(}
    \AttributeTok{values =}\NormalTok{ fcols      }
\NormalTok{  )}

\NormalTok{ppp.plot}
\end{Highlighting}
\end{Shaded}

\includegraphics{unsupervised-learning_files/figure-latex/unnamed-chunk-24-1.pdf}

\begin{Shaded}
\begin{Highlighting}[]
\FunctionTok{ggplotly}\NormalTok{(ppp.plot, }\AttributeTok{tooltip =} \StringTok{"text"}\NormalTok{)}
\end{Highlighting}
\end{Shaded}

\includegraphics{unsupervised-learning_files/figure-latex/unnamed-chunk-25-1.pdf}

\hypertarget{nilai-harian-x-ppp}{%
\subsubsection{Nilai harian x ppp}\label{nilai-harian-x-ppp}}

\begin{Shaded}
\begin{Highlighting}[]
\NormalTok{hxp.plot }\OtherTok{\textless{}{-}} 
  \FunctionTok{ggplot}\NormalTok{(data, }
         \FunctionTok{aes}\NormalTok{(}\AttributeTok{x =}\NormalTok{ nilai\_ppp, }
             \AttributeTok{y =}\NormalTok{ nilai\_ph))}\SpecialCharTok{+}
  \FunctionTok{geom\_jitter}\NormalTok{(}\FunctionTok{aes}\NormalTok{(}\AttributeTok{col =}\NormalTok{ kelas,}
                  \AttributeTok{text =}\NormalTok{ exp),}
              \AttributeTok{width =} \FloatTok{0.3}\NormalTok{, }
              \AttributeTok{height =} \FloatTok{0.3}\NormalTok{, }
              \AttributeTok{alpha =} \FloatTok{0.7}\NormalTok{)}\SpecialCharTok{+}
  \FunctionTok{theme\_minimal}\NormalTok{()}\SpecialCharTok{+}
  \FunctionTok{labs}\NormalTok{(}\AttributeTok{title =} \StringTok{"Hubungan Nilai PPP dan Penilaian Harian"}\NormalTok{,}
       \AttributeTok{x =} \StringTok{"Nilai PPP"}\NormalTok{,}
       \AttributeTok{y =} \StringTok{"Penilaian Harian"}\NormalTok{)}\SpecialCharTok{+}
    \FunctionTok{scale\_color\_manual}\NormalTok{(}
    \AttributeTok{values =}\NormalTok{ fcols      }
\NormalTok{  )}\SpecialCharTok{+}
  \FunctionTok{theme}\NormalTok{(}\AttributeTok{legend.position =} \StringTok{"none"}\NormalTok{,}
        \AttributeTok{axis.title.y =} \FunctionTok{element\_blank}\NormalTok{())}

  

\NormalTok{hxp.plot}
\end{Highlighting}
\end{Shaded}

\includegraphics{unsupervised-learning_files/figure-latex/unnamed-chunk-26-1.pdf}

\begin{Shaded}
\begin{Highlighting}[]
\FunctionTok{ggplotly}\NormalTok{(hxp.plot, }\AttributeTok{tooltip =} \StringTok{"text"}\NormalTok{)}
\end{Highlighting}
\end{Shaded}

\includegraphics{unsupervised-learning_files/figure-latex/unnamed-chunk-27-1.pdf}

\hypertarget{korelasi-nilai-penilaian-ranah-keterampilan-dan-penilaian-harian}{%
\paragraph{Korelasi Nilai Penilaian Ranah Keterampilan dan Penilaian
Harian}\label{korelasi-nilai-penilaian-ranah-keterampilan-dan-penilaian-harian}}

\begin{Shaded}
\begin{Highlighting}[]
\NormalTok{corp3ph.kelas.materi }\OtherTok{\textless{}{-}}\NormalTok{ data }\SpecialCharTok{\%\textgreater{}\%} 
  \FunctionTok{group\_by}\NormalTok{(kelas, materi) }\SpecialCharTok{\%\textgreater{}\%} 
  \FunctionTok{summarize}\NormalTok{(}\AttributeTok{correlation =} \FunctionTok{cor}\NormalTok{(nilai\_ppp, nilai\_ph))}
\NormalTok{corp3ph.kelas.materi}
\end{Highlighting}
\end{Shaded}

\begin{verbatim}
## # A tibble: 12 x 3
## # Groups:   kelas [3]
##    kelas    materi            correlation
##    <fct>    <fct>                   <dbl>
##  1 XI IPS 1 Drama                   1    
##  2 XI IPS 1 Karya Ilmiah            0.806
##  3 XI IPS 1 Proposal Kegiatan       0.813
##  4 XI IPS 1 Resensi                 0.918
##  5 XI IPS 2 Drama                   1    
##  6 XI IPS 2 Karya Ilmiah            0.805
##  7 XI IPS 2 Proposal Kegiatan       0.881
##  8 XI IPS 2 Resensi                 0.817
##  9 XI MIPA  Drama                   1    
## 10 XI MIPA  Karya Ilmiah            0.771
## 11 XI MIPA  Proposal Kegiatan       0.819
## 12 XI MIPA  Resensi                 0.882
\end{verbatim}

\begin{Shaded}
\begin{Highlighting}[]
\NormalTok{corp3ph.kelas.materi }\SpecialCharTok{\%\textgreater{}\%} \FunctionTok{slice\_max}\NormalTok{(correlation)}
\end{Highlighting}
\end{Shaded}

\begin{verbatim}
## # A tibble: 3 x 3
## # Groups:   kelas [3]
##   kelas    materi correlation
##   <fct>    <fct>        <dbl>
## 1 XI IPS 1 Drama            1
## 2 XI IPS 2 Drama            1
## 3 XI MIPA  Drama            1
\end{verbatim}

\begin{Shaded}
\begin{Highlighting}[]
\NormalTok{corp3ph.kelas.materi }\SpecialCharTok{\%\textgreater{}\%} \FunctionTok{slice\_min}\NormalTok{(correlation)}
\end{Highlighting}
\end{Shaded}

\begin{verbatim}
## # A tibble: 3 x 3
## # Groups:   kelas [3]
##   kelas    materi       correlation
##   <fct>    <fct>              <dbl>
## 1 XI IPS 1 Karya Ilmiah       0.806
## 2 XI IPS 2 Karya Ilmiah       0.805
## 3 XI MIPA  Karya Ilmiah       0.771
\end{verbatim}

\begin{Shaded}
\begin{Highlighting}[]
\NormalTok{cor.plot.k.m }\OtherTok{\textless{}{-}} 
\FunctionTok{ggplot}\NormalTok{(corp3ph.kelas.materi, }
       \FunctionTok{aes}\NormalTok{(}\AttributeTok{x =}\NormalTok{ materi, }
           \AttributeTok{y =}\NormalTok{ correlation,}
           \AttributeTok{text =}\NormalTok{ kelas))}\SpecialCharTok{+}
  \FunctionTok{geom\_col}\NormalTok{(}\FunctionTok{aes}\NormalTok{(}\AttributeTok{fill =}\NormalTok{ kelas), }
           \AttributeTok{position =} \FunctionTok{position\_dodge}\NormalTok{()}
\NormalTok{           )}\SpecialCharTok{+}
    \FunctionTok{theme\_minimal}\NormalTok{()}\SpecialCharTok{+}
  \FunctionTok{labs}\NormalTok{(}\AttributeTok{title =} \StringTok{"Korelasi Materi dan Nilai Harian"}\NormalTok{,}
       \AttributeTok{x =} \StringTok{"Materi"}\NormalTok{)}\SpecialCharTok{+}
    \FunctionTok{scale\_fill\_manual}\NormalTok{(}
    \AttributeTok{values =}\NormalTok{ fcols      }
\NormalTok{  )}\SpecialCharTok{+}
  \FunctionTok{theme}\NormalTok{(}\AttributeTok{legend.position =} \StringTok{"none"}\NormalTok{,}
        \AttributeTok{axis.title.y =} \FunctionTok{element\_blank}\NormalTok{())}

\NormalTok{cor.plot.k.m}
\end{Highlighting}
\end{Shaded}

\includegraphics{unsupervised-learning_files/figure-latex/unnamed-chunk-30-1.pdf}

\begin{Shaded}
\begin{Highlighting}[]
\FunctionTok{ggplotly}\NormalTok{(cor.plot.k.m, }\AttributeTok{tooltip =} \StringTok{"text"}\NormalTok{)}
\end{Highlighting}
\end{Shaded}

\includegraphics{unsupervised-learning_files/figure-latex/unnamed-chunk-31-1.pdf}

\hypertarget{korelasi-penilaian-ranah-keterampilan-dan-penilaian-harian}{%
\paragraph{Korelasi Penilaian Ranah Keterampilan dan Penilaian
Harian}\label{korelasi-penilaian-ranah-keterampilan-dan-penilaian-harian}}

\begin{Shaded}
\begin{Highlighting}[]
\NormalTok{corp3ph.kelas.p3 }\OtherTok{\textless{}{-}}\NormalTok{ data }\SpecialCharTok{\%\textgreater{}\%} 
  \FunctionTok{group\_by}\NormalTok{(kelas, ppp) }\SpecialCharTok{\%\textgreater{}\%} 
  \FunctionTok{summarize}\NormalTok{(}\AttributeTok{correlation =} \FunctionTok{cor}\NormalTok{(nilai\_ppp, nilai\_ph))}
\NormalTok{corp3ph.kelas.p3}
\end{Highlighting}
\end{Shaded}

\begin{verbatim}
## # A tibble: 9 x 3
## # Groups:   kelas [3]
##   kelas    ppp        correlation
##   <fct>    <chr>            <dbl>
## 1 XI IPS 1 Portofolio       0.921
## 2 XI IPS 1 Praktek          0.732
## 3 XI IPS 1 Proyek           0.960
## 4 XI IPS 2 Portofolio       0.902
## 5 XI IPS 2 Praktek          0.731
## 6 XI IPS 2 Proyek           0.962
## 7 XI MIPA  Portofolio       0.937
## 8 XI MIPA  Praktek          0.621
## 9 XI MIPA  Proyek           0.949
\end{verbatim}

\begin{Shaded}
\begin{Highlighting}[]
\NormalTok{corp3ph.kelas.p3 }\SpecialCharTok{\%\textgreater{}\%} \FunctionTok{slice\_max}\NormalTok{(correlation)}
\end{Highlighting}
\end{Shaded}

\begin{verbatim}
## # A tibble: 3 x 3
## # Groups:   kelas [3]
##   kelas    ppp    correlation
##   <fct>    <chr>        <dbl>
## 1 XI IPS 1 Proyek       0.960
## 2 XI IPS 2 Proyek       0.962
## 3 XI MIPA  Proyek       0.949
\end{verbatim}

\begin{Shaded}
\begin{Highlighting}[]
\NormalTok{corp3ph.kelas.p3 }\SpecialCharTok{\%\textgreater{}\%} \FunctionTok{slice\_min}\NormalTok{(correlation)}
\end{Highlighting}
\end{Shaded}

\begin{verbatim}
## # A tibble: 3 x 3
## # Groups:   kelas [3]
##   kelas    ppp     correlation
##   <fct>    <chr>         <dbl>
## 1 XI IPS 1 Praktek       0.732
## 2 XI IPS 2 Praktek       0.731
## 3 XI MIPA  Praktek       0.621
\end{verbatim}

\begin{Shaded}
\begin{Highlighting}[]
\NormalTok{cor.plot.k.p }\OtherTok{\textless{}{-}} 
\FunctionTok{ggplot}\NormalTok{(corp3ph.kelas.p3, }\FunctionTok{aes}\NormalTok{(ppp, correlation))}\SpecialCharTok{+}
    \FunctionTok{geom\_col}\NormalTok{(}\FunctionTok{aes}\NormalTok{(}\AttributeTok{fill =}\NormalTok{ kelas, }\AttributeTok{text =}\NormalTok{ kelas), }
           \AttributeTok{position =} \FunctionTok{position\_dodge}\NormalTok{()}
\NormalTok{           )}\SpecialCharTok{+}
    \FunctionTok{theme\_minimal}\NormalTok{()}\SpecialCharTok{+}
  \FunctionTok{labs}\NormalTok{(}\AttributeTok{title =} \StringTok{"Korelasi Penilaian Ranah Keterampilan dan Penilaian Harian"}\NormalTok{,}
       \AttributeTok{subtitle =} \StringTok{"Berdasarkan Proyek/Praktek/Portfolio"}\NormalTok{,}
       \AttributeTok{x =} \StringTok{"Ranah Keterampilan"}\NormalTok{)}\SpecialCharTok{+}
    \FunctionTok{scale\_fill\_manual}\NormalTok{(}
    \AttributeTok{values =}\NormalTok{ fcols      }
\NormalTok{  )}\SpecialCharTok{+}
  \FunctionTok{theme}\NormalTok{(}\AttributeTok{legend.position =} \StringTok{"none"}\NormalTok{,}
        \AttributeTok{axis.title.y =} \FunctionTok{element\_blank}\NormalTok{())}

\NormalTok{cor.plot.k.p}
\end{Highlighting}
\end{Shaded}

\includegraphics{unsupervised-learning_files/figure-latex/unnamed-chunk-34-1.pdf}

\begin{Shaded}
\begin{Highlighting}[]
\FunctionTok{ggplotly}\NormalTok{(cor.plot.k.p, }\AttributeTok{tooltip =} \StringTok{"text"}\NormalTok{)}
\end{Highlighting}
\end{Shaded}

\includegraphics{unsupervised-learning_files/figure-latex/unnamed-chunk-35-1.pdf}

\hypertarget{perbandingan-nilai-semesteran-antar-kelas}{%
\subsubsection{Perbandingan Nilai Semesteran Antar
Kelas}\label{perbandingan-nilai-semesteran-antar-kelas}}

\begin{Shaded}
\begin{Highlighting}[]
\NormalTok{plot.semester.k }\OtherTok{\textless{}{-}}
\FunctionTok{ggplot}\NormalTok{(data, }\FunctionTok{aes}\NormalTok{(}\AttributeTok{x =}\NormalTok{ nilai, }\AttributeTok{text =}\NormalTok{ exp))}\SpecialCharTok{+}
  \FunctionTok{geom\_histogram}\NormalTok{(}\FunctionTok{aes}\NormalTok{(}\AttributeTok{fill =}\NormalTok{ kelas),}
                 \AttributeTok{binwidth =} \DecValTok{1}\NormalTok{)}\SpecialCharTok{+}
  \FunctionTok{facet\_wrap}\NormalTok{(}\SpecialCharTok{\textasciitilde{}}\NormalTok{penilaian\_s }\SpecialCharTok{+}\NormalTok{ kelas)}\SpecialCharTok{+}
  \FunctionTok{theme\_minimal}\NormalTok{()}\SpecialCharTok{+}
  \FunctionTok{theme}\NormalTok{(}\AttributeTok{legend.position =} \StringTok{"none"}\NormalTok{)}\SpecialCharTok{+}
  \FunctionTok{labs}\NormalTok{(}\AttributeTok{title =} \StringTok{"Perbandingan Sebaran Nilai Semesteran Antar Kelas"}\NormalTok{,}
       \AttributeTok{x =} \StringTok{"Nilai"}\NormalTok{,}
       \AttributeTok{y =} \StringTok{"Jumlah Siswa"}\NormalTok{)}\SpecialCharTok{+}
  \FunctionTok{scale\_fill\_manual}\NormalTok{(}
    \AttributeTok{values =}\NormalTok{ fcols      }
\NormalTok{  )}
\NormalTok{plot.semester.k}
\end{Highlighting}
\end{Shaded}

\includegraphics{unsupervised-learning_files/figure-latex/unnamed-chunk-36-1.pdf}

\begin{Shaded}
\begin{Highlighting}[]
\FunctionTok{ggplotly}\NormalTok{(plot.semester.k, }\AttributeTok{tooltip =} \StringTok{"text"}\NormalTok{)}
\end{Highlighting}
\end{Shaded}

\includegraphics{unsupervised-learning_files/figure-latex/unnamed-chunk-37-1.pdf}

\hypertarget{korelasi-nilai-harian-dan-nilai-semesteran}{%
\paragraph{Korelasi Nilai Harian dan Nilai
Semesteran}\label{korelasi-nilai-harian-dan-nilai-semesteran}}

\begin{Shaded}
\begin{Highlighting}[]
\NormalTok{k.kelas.ph.s }\OtherTok{\textless{}{-}} 
\NormalTok{  data }\SpecialCharTok{\%\textgreater{}\%} 
  \FunctionTok{group\_by}\NormalTok{(kelas, materi) }\SpecialCharTok{\%\textgreater{}\%} 
  \FunctionTok{summarize}\NormalTok{(}\AttributeTok{correlation =} \FunctionTok{cor}\NormalTok{(nilai\_ph, nilai))}

\NormalTok{k.kelas.ph.s}
\end{Highlighting}
\end{Shaded}

\begin{verbatim}
## # A tibble: 12 x 3
## # Groups:   kelas [3]
##    kelas    materi            correlation
##    <fct>    <fct>                   <dbl>
##  1 XI IPS 1 Drama                   0.557
##  2 XI IPS 1 Karya Ilmiah            0.396
##  3 XI IPS 1 Proposal Kegiatan       0.481
##  4 XI IPS 1 Resensi                 0.686
##  5 XI IPS 2 Drama                   0.285
##  6 XI IPS 2 Karya Ilmiah            0.182
##  7 XI IPS 2 Proposal Kegiatan       0.588
##  8 XI IPS 2 Resensi                 0.400
##  9 XI MIPA  Drama                   0.212
## 10 XI MIPA  Karya Ilmiah            0.338
## 11 XI MIPA  Proposal Kegiatan       0.330
## 12 XI MIPA  Resensi                 0.527
\end{verbatim}

\begin{Shaded}
\begin{Highlighting}[]
\NormalTok{bcols }\OtherTok{\textless{}{-}} \FunctionTok{c}\NormalTok{(}\StringTok{"Drama"} \OtherTok{=} \StringTok{"\#C6E0FF"}\NormalTok{, }
           \StringTok{"Karya Ilmiah"} \OtherTok{=} \StringTok{"\#579A9E"}\NormalTok{,}
           \StringTok{"Proposal Kegiatan"} \OtherTok{=} \StringTok{"\#3292C3"}\NormalTok{,}
           \StringTok{"Resensi"} \OtherTok{=} \StringTok{"\#BCAB79"}\NormalTok{)}

\NormalTok{plot.k.h.s }\OtherTok{\textless{}{-}}
\FunctionTok{ggplot}\NormalTok{(k.kelas.ph.s, }\FunctionTok{aes}\NormalTok{(}\AttributeTok{x =}\NormalTok{ kelas, }
                         \AttributeTok{y =}\NormalTok{ correlation,}
                         \AttributeTok{fill =}\NormalTok{ materi,}
                         \AttributeTok{text =}\NormalTok{ materi))}\SpecialCharTok{+}
  \FunctionTok{geom\_col}\NormalTok{(}\AttributeTok{position =} \FunctionTok{position\_dodge}\NormalTok{())}\SpecialCharTok{+}
  \FunctionTok{scale\_fill\_manual}\NormalTok{(}
    \AttributeTok{values =}\NormalTok{ bcols}
\NormalTok{  )}\SpecialCharTok{+}
  \FunctionTok{theme\_minimal}\NormalTok{()}\SpecialCharTok{+}
  \FunctionTok{theme}\NormalTok{(}\AttributeTok{legend.position =} \StringTok{"none"}\NormalTok{,}
        \AttributeTok{axis.title.x =} \FunctionTok{element\_blank}\NormalTok{(),}
        \AttributeTok{axis.title.y =} \FunctionTok{element\_blank}\NormalTok{())}\SpecialCharTok{+}
  \FunctionTok{labs}\NormalTok{(}\AttributeTok{title =} \StringTok{"Korelasi Nilai Harian dan Semesteran"}\NormalTok{,}
       \AttributeTok{subtitle =} \StringTok{"Berdasarkan Kelas dan Materi"}\NormalTok{)}
  

\NormalTok{plot.k.h.s}
\end{Highlighting}
\end{Shaded}

\includegraphics{unsupervised-learning_files/figure-latex/unnamed-chunk-39-1.pdf}

\begin{Shaded}
\begin{Highlighting}[]
\FunctionTok{ggplotly}\NormalTok{(plot.k.h.s, }\AttributeTok{tooltip =} \StringTok{"text"}\NormalTok{)}
\end{Highlighting}
\end{Shaded}

\includegraphics{unsupervised-learning_files/figure-latex/unnamed-chunk-40-1.pdf}

\hypertarget{korelasi-nilai-ranah-keterampilan-dan-nilai-semesteran}{%
\paragraph{Korelasi Nilai Ranah Keterampilan dan Nilai
Semesteran}\label{korelasi-nilai-ranah-keterampilan-dan-nilai-semesteran}}

\begin{Shaded}
\begin{Highlighting}[]
\NormalTok{k.kelas.ppp.s }\OtherTok{\textless{}{-}} 
\NormalTok{  data }\SpecialCharTok{\%\textgreater{}\%} 
  \FunctionTok{group\_by}\NormalTok{(ppp, kelas) }\SpecialCharTok{\%\textgreater{}\%} 
  \FunctionTok{summarize}\NormalTok{(}\AttributeTok{correlation =} \FunctionTok{cor}\NormalTok{(nilai\_ppp, nilai)) }\SpecialCharTok{\%\textgreater{}\%} 
  \FunctionTok{arrange}\NormalTok{(}\FunctionTok{desc}\NormalTok{(correlation))}

\NormalTok{k.kelas.ppp.s}
\end{Highlighting}
\end{Shaded}

\begin{verbatim}
## # A tibble: 9 x 3
## # Groups:   ppp [3]
##   ppp        kelas    correlation
##   <chr>      <fct>          <dbl>
## 1 Proyek     XI IPS 1       0.512
## 2 Portofolio XI IPS 1       0.469
## 3 Praktek    XI IPS 1       0.363
## 4 Proyek     XI MIPA        0.344
## 5 Praktek    XI IPS 2       0.330
## 6 Portofolio XI MIPA        0.304
## 7 Proyek     XI IPS 2       0.294
## 8 Portofolio XI IPS 2       0.253
## 9 Praktek    XI MIPA        0.202
\end{verbatim}

\begin{Shaded}
\begin{Highlighting}[]
\NormalTok{plot.ppp.k.s }\OtherTok{\textless{}{-}}
\FunctionTok{ggplot}\NormalTok{(k.kelas.ppp.s, }\FunctionTok{aes}\NormalTok{(}\AttributeTok{x =}\NormalTok{ ppp, }
                         \AttributeTok{y =}\NormalTok{ correlation,}
                         \AttributeTok{fill =}\NormalTok{ kelas,}
                         \AttributeTok{text =}\NormalTok{ kelas))}\SpecialCharTok{+}
  \FunctionTok{geom\_col}\NormalTok{(}\AttributeTok{position =} \FunctionTok{position\_dodge}\NormalTok{())}\SpecialCharTok{+}
  \FunctionTok{scale\_fill\_manual}\NormalTok{(}
    \AttributeTok{values =}\NormalTok{ fcols}
\NormalTok{  )}\SpecialCharTok{+}
  \FunctionTok{theme\_minimal}\NormalTok{()}\SpecialCharTok{+}
  \FunctionTok{theme}\NormalTok{(}\AttributeTok{legend.position =} \StringTok{"none"}\NormalTok{,}
        \AttributeTok{axis.title.x =} \FunctionTok{element\_blank}\NormalTok{(),}
        \AttributeTok{axis.title.y =} \FunctionTok{element\_blank}\NormalTok{())}\SpecialCharTok{+}
  \FunctionTok{labs}\NormalTok{(}\AttributeTok{title =} \StringTok{"Korelasi Penilaian Ranah Keterampilan dan Nilai Semesteran"}\NormalTok{,}
       \AttributeTok{subtitle =} \StringTok{"Berdasarkan Kelas"}\NormalTok{)}
  

\NormalTok{plot.ppp.k.s}
\end{Highlighting}
\end{Shaded}

\includegraphics{unsupervised-learning_files/figure-latex/unnamed-chunk-42-1.pdf}

\begin{Shaded}
\begin{Highlighting}[]
\FunctionTok{ggplotly}\NormalTok{(plot.ppp.k.s, }\AttributeTok{tooltip =} \StringTok{"text"}\NormalTok{)}
\end{Highlighting}
\end{Shaded}

\includegraphics{unsupervised-learning_files/figure-latex/unnamed-chunk-43-1.pdf}

\hypertarget{perbandingan-ranah-keterampilan-hingga-semesteran}{%
\subsubsection{Perbandingan Ranah Keterampilan Hingga
Semesteran}\label{perbandingan-ranah-keterampilan-hingga-semesteran}}

\begin{Shaded}
\begin{Highlighting}[]
\NormalTok{data\_edit\_col }\OtherTok{\textless{}{-}} 
\NormalTok{  data }\SpecialCharTok{\%\textgreater{}\%} 
  \FunctionTok{mutate}\NormalTok{(}\StringTok{"Nilai Ranah Keterampilan"} \OtherTok{=}\NormalTok{ nilai\_ppp,}
         \StringTok{"Nilai Harian"} \OtherTok{=}\NormalTok{ nilai\_ph,}
         \StringTok{"Nilai Semesteran"} \OtherTok{=}\NormalTok{ nilai,}
         \StringTok{"Kelas"} \OtherTok{=}\NormalTok{ kelas)}

\NormalTok{p.par.nilai }\OtherTok{\textless{}{-}} 
\FunctionTok{ggparcoord}\NormalTok{(data\_edit\_col, }
           \AttributeTok{columns =} \FunctionTok{c}\NormalTok{(}\StringTok{"Nilai Ranah Keterampilan"}\NormalTok{, }
                       \StringTok{"Nilai Harian"}\NormalTok{, }
                       \StringTok{"Nilai Semesteran"}\NormalTok{),}
           \AttributeTok{groupColumn =} \StringTok{"Kelas"}\NormalTok{,}
           \AttributeTok{scale =} \StringTok{"uniminmax"}\NormalTok{,}
           \AttributeTok{order =} \FunctionTok{c}\NormalTok{(}\DecValTok{11}\NormalTok{,}\DecValTok{12}\NormalTok{,}\DecValTok{13}\NormalTok{),}
           \AttributeTok{showPoints =} \ConstantTok{TRUE}\NormalTok{,}
           \AttributeTok{splineFactor =} \DecValTok{4}\NormalTok{,}
           \AttributeTok{title =} \StringTok{"Paralel Plot Nilai Bahasa Indonesia"}\NormalTok{,}
           \AttributeTok{alphaLines =} \FloatTok{0.6}\NormalTok{)}\SpecialCharTok{+} 
  \FunctionTok{scale\_color\_manual}\NormalTok{(}\AttributeTok{values =}\NormalTok{ fcols) }\SpecialCharTok{+}
  \FunctionTok{theme}\NormalTok{(}
    \AttributeTok{axis.title.x =} \FunctionTok{element\_blank}\NormalTok{()}
\NormalTok{  )}\SpecialCharTok{+}
  \FunctionTok{theme\_minimal}\NormalTok{()}\SpecialCharTok{+}
  \FunctionTok{labs}\NormalTok{(}\AttributeTok{y =} \StringTok{"Scaled with UniMinMax"}\NormalTok{,}
       \AttributeTok{col =} \StringTok{"Kelas"}\NormalTok{)}
\end{Highlighting}
\end{Shaded}

\begin{Shaded}
\begin{Highlighting}[]
\FunctionTok{ggplotly}\NormalTok{(p.par.nilai, }\AttributeTok{tooltip =} \StringTok{"Kelas"}\NormalTok{)}
\end{Highlighting}
\end{Shaded}

\includegraphics{unsupervised-learning_files/figure-latex/unnamed-chunk-45-1.pdf}

\hypertarget{perbandingan-korelasi}{%
\paragraph{Perbandingan Korelasi}\label{perbandingan-korelasi}}

\begin{Shaded}
\begin{Highlighting}[]
\NormalTok{s.p3.h.s }\OtherTok{\textless{}{-}}
\NormalTok{data }\SpecialCharTok{\%\textgreater{}\%} 
  \FunctionTok{group\_by}\NormalTok{(ppp, penilaian\_h, penilaian\_s) }\SpecialCharTok{\%\textgreater{}\%} 
  \FunctionTok{summarise}\NormalTok{(}\AttributeTok{kor\_ppp\_ph =} \FunctionTok{cor}\NormalTok{(nilai\_ppp, nilai\_ph),}
            \AttributeTok{kor\_ph\_s =} \FunctionTok{cor}\NormalTok{(nilai\_ph, nilai))}

\NormalTok{s.p3.h.s}
\end{Highlighting}
\end{Shaded}

\begin{verbatim}
## # A tibble: 24 x 5
## # Groups:   ppp, penilaian_h [12]
##    ppp        penilaian_h penilaian_s kor_ppp_ph kor_ph_s
##    <chr>      <fct>       <fct>            <dbl>    <dbl>
##  1 Portofolio Harian 1    PAT              0.930    0.437
##  2 Portofolio Harian 1    PHBO             0.930    0.490
##  3 Portofolio Harian 2    PAT              1.00     0.257
##  4 Portofolio Harian 2    PHBO             1.00     0.377
##  5 Portofolio Harian 3    PAT              0.785    0.533
##  6 Portofolio Harian 3    PHBO             0.785    0.583
##  7 Portofolio Harian 4    PAT              1        0.364
##  8 Portofolio Harian 4    PHBO             1        0.381
##  9 Praktek    Harian 1    PAT              0.653    0.437
## 10 Praktek    Harian 1    PHBO             0.653    0.490
## # ... with 14 more rows
## # i Use `print(n = ...)` to see more rows
\end{verbatim}

Impute missing values.

\begin{Shaded}
\begin{Highlighting}[]
\NormalTok{imputed.s.p3.h.s }\OtherTok{\textless{}{-}}
  \FunctionTok{recipe}\NormalTok{(}\SpecialCharTok{\textasciitilde{}}\NormalTok{ ., }\AttributeTok{data =}\NormalTok{ s.p3.h.s) }\SpecialCharTok{\%\textgreater{}\%}
  \FunctionTok{step\_impute\_bag}\NormalTok{(kor\_ppp\_ph) }\SpecialCharTok{\%\textgreater{}\%} 
  \FunctionTok{prep}\NormalTok{(s.p3.h.s)}
\NormalTok{imputed.s.p3.h.s }
\end{Highlighting}
\end{Shaded}

\begin{verbatim}
## Recipe
## 
## Inputs:
## 
##       role #variables
##  predictor          5
## 
## Training data contained 24 data points and 2 incomplete rows. 
## 
## Operations:
## 
## Bagged tree imputation for kor_ppp_ph [trained]
\end{verbatim}

\begin{Shaded}
\begin{Highlighting}[]
\NormalTok{df.imputed.s.p3.h.s }\OtherTok{\textless{}{-}}
  \FunctionTok{bake}\NormalTok{(imputed.s.p3.h.s, }\AttributeTok{new\_data =}\NormalTok{ s.p3.h.s) }
\NormalTok{df.imputed.s.p3.h.s}
\end{Highlighting}
\end{Shaded}

\begin{verbatim}
## # A tibble: 24 x 5
## # Groups:   ppp, penilaian_h [12]
##    ppp        penilaian_h penilaian_s kor_ppp_ph kor_ph_s
##    <fct>      <fct>       <fct>            <dbl>    <dbl>
##  1 Portofolio Harian 1    PAT              0.930    0.437
##  2 Portofolio Harian 1    PHBO             0.930    0.490
##  3 Portofolio Harian 2    PAT              1.00     0.257
##  4 Portofolio Harian 2    PHBO             1.00     0.377
##  5 Portofolio Harian 3    PAT              0.785    0.533
##  6 Portofolio Harian 3    PHBO             0.785    0.583
##  7 Portofolio Harian 4    PAT              1        0.364
##  8 Portofolio Harian 4    PHBO             1        0.381
##  9 Praktek    Harian 1    PAT              0.653    0.437
## 10 Praktek    Harian 1    PHBO             0.653    0.490
## # ... with 14 more rows
## # i Use `print(n = ...)` to see more rows
\end{verbatim}

\begin{Shaded}
\begin{Highlighting}[]
\FunctionTok{anyNA}\NormalTok{(df.imputed.s.p3.h.s)}
\end{Highlighting}
\end{Shaded}

\begin{verbatim}
## [1] FALSE
\end{verbatim}

\begin{Shaded}
\begin{Highlighting}[]
\NormalTok{pcols }\OtherTok{=} \FunctionTok{c}\NormalTok{(}
  \StringTok{"Portofolio"} \OtherTok{=} \StringTok{"\#BEEF9E"}\NormalTok{,}
  \StringTok{"Praktek"} \OtherTok{=} \StringTok{"\#3292C3"}\NormalTok{,}
  \StringTok{"Proyek"} \OtherTok{=} \StringTok{"\#BCAB79"}
\NormalTok{)}
  
\NormalTok{p.dens }\OtherTok{\textless{}{-}}
\FunctionTok{ggplot}\NormalTok{(df.imputed.s.p3.h.s, }\FunctionTok{aes}\NormalTok{(kor\_ppp\_ph, kor\_ph\_s)) }\SpecialCharTok{+}
  \FunctionTok{stat\_density2d}\NormalTok{(}\AttributeTok{geom=}\StringTok{"tile"}\NormalTok{, }\FunctionTok{aes}\NormalTok{(}\AttributeTok{fill =}\NormalTok{ ..density..), }\AttributeTok{contour =} \ConstantTok{FALSE}\NormalTok{) }\SpecialCharTok{+} 
  \FunctionTok{geom\_point}\NormalTok{(}\AttributeTok{size =} \DecValTok{6}\NormalTok{, }\AttributeTok{color =} \StringTok{"white"}\NormalTok{)}\SpecialCharTok{+}
  \FunctionTok{geom\_point}\NormalTok{(}\FunctionTok{aes}\NormalTok{(}\AttributeTok{col=}\NormalTok{ppp, }\AttributeTok{text =}\NormalTok{ penilaian\_s), }\AttributeTok{size =} \DecValTok{3}\NormalTok{)}\SpecialCharTok{+}
  \FunctionTok{labs}\NormalTok{(}\AttributeTok{title =} \StringTok{"Hubungan Nilai Korelasi"}\NormalTok{,}
       \AttributeTok{x =} \StringTok{"Korelasi Nilai Ranah Keterampilan dan Nilai Harian"}\NormalTok{,}
       \AttributeTok{y =} \StringTok{"Korelasi Nilai Harian dan Nilai Semesteran"}\NormalTok{,}
       \AttributeTok{color =} \StringTok{"Keterampilan"}\NormalTok{,}
       \AttributeTok{fill =} \StringTok{"Density"}
\NormalTok{       )}\SpecialCharTok{+}
  \FunctionTok{theme\_minimal}\NormalTok{()}\SpecialCharTok{+}
  \FunctionTok{scale\_color\_manual}\NormalTok{(}\AttributeTok{values =}\NormalTok{ pcols)}
  

\NormalTok{p.dens }
\end{Highlighting}
\end{Shaded}

\includegraphics{unsupervised-learning_files/figure-latex/unnamed-chunk-49-1.pdf}

\begin{Shaded}
\begin{Highlighting}[]
\FunctionTok{ggplotly}\NormalTok{(p.dens, }\AttributeTok{tooltip =} \StringTok{"text"}\NormalTok{)}
\end{Highlighting}
\end{Shaded}

\includegraphics{unsupervised-learning_files/figure-latex/unnamed-chunk-50-1.pdf}

\hypertarget{uji-korelasi}{%
\paragraph{Uji Korelasi}\label{uji-korelasi}}

\begin{Shaded}
\begin{Highlighting}[]
\NormalTok{uk.ppp.s }\OtherTok{\textless{}{-}}
\NormalTok{data }\SpecialCharTok{\%\textgreater{}\%} 
  \FunctionTok{select}\NormalTok{(kelas, nilai\_ppp, nilai) }\SpecialCharTok{\%\textgreater{}\%} 
  \FunctionTok{nest}\NormalTok{(}\AttributeTok{data =} \FunctionTok{c}\NormalTok{(nilai\_ppp, nilai)) }\SpecialCharTok{\%\textgreater{}\%} 
  \FunctionTok{mutate}\NormalTok{(}
    \AttributeTok{test =} \FunctionTok{map}\NormalTok{(data, }\SpecialCharTok{\textasciitilde{}} \FunctionTok{cor.test}\NormalTok{(.}\SpecialCharTok{$}\NormalTok{nilai\_ppp,}
\NormalTok{                               .}\SpecialCharTok{$}\NormalTok{nilai)),}
    \AttributeTok{tidied =} \FunctionTok{map}\NormalTok{(test, tidy)}
\NormalTok{  ) }\SpecialCharTok{\%\textgreater{}\%} 
  \FunctionTok{unnest}\NormalTok{(}\AttributeTok{cols =}\NormalTok{ tidied) }\SpecialCharTok{\%\textgreater{}\%} 
  \FunctionTok{select}\NormalTok{(}\SpecialCharTok{{-}}\NormalTok{data, }\SpecialCharTok{{-}}\NormalTok{test)}
\NormalTok{uk.ppp.s}
\end{Highlighting}
\end{Shaded}

\begin{verbatim}
## # A tibble: 3 x 9
##   kelas    estimate statistic  p.value parameter conf.low conf.~1 method alter~2
##   <fct>       <dbl>     <dbl>    <dbl>     <int>    <dbl>   <dbl> <chr>  <chr>  
## 1 XI IPS 1    0.451     12.6  1.25e-32       622    0.386   0.512 Pears~ two.si~
## 2 XI IPS 2    0.289      7.23 1.58e-12       574    0.212   0.362 Pears~ two.si~
## 3 XI MIPA     0.288      8.31 4.31e-16       766    0.221   0.351 Pears~ two.si~
## # ... with abbreviated variable names 1: conf.high, 2: alternative
\end{verbatim}

\begin{Shaded}
\begin{Highlighting}[]
\NormalTok{uk.ph.s }\OtherTok{\textless{}{-}}
\NormalTok{data }\SpecialCharTok{\%\textgreater{}\%} 
  \FunctionTok{select}\NormalTok{(kelas, nilai\_ph, nilai) }\SpecialCharTok{\%\textgreater{}\%} 
  \FunctionTok{nest}\NormalTok{(}\AttributeTok{data =} \FunctionTok{c}\NormalTok{(nilai\_ph, nilai)) }\SpecialCharTok{\%\textgreater{}\%} 
  \FunctionTok{mutate}\NormalTok{(}
    \AttributeTok{test =} \FunctionTok{map}\NormalTok{(data, }\SpecialCharTok{\textasciitilde{}} \FunctionTok{cor.test}\NormalTok{(.}\SpecialCharTok{$}\NormalTok{nilai\_ph,}
\NormalTok{                               .}\SpecialCharTok{$}\NormalTok{nilai)),}
    \AttributeTok{tidied =} \FunctionTok{map}\NormalTok{(test, tidy)}
\NormalTok{  ) }\SpecialCharTok{\%\textgreater{}\%} 
  \FunctionTok{unnest}\NormalTok{(}\AttributeTok{cols =}\NormalTok{ tidied) }\SpecialCharTok{\%\textgreater{}\%} 
  \FunctionTok{select}\NormalTok{(}\SpecialCharTok{{-}}\NormalTok{data, }\SpecialCharTok{{-}}\NormalTok{test)}
\end{Highlighting}
\end{Shaded}

\begin{Shaded}
\begin{Highlighting}[]
\NormalTok{uk.ph.s}
\end{Highlighting}
\end{Shaded}

\begin{verbatim}
## # A tibble: 3 x 9
##   kelas    estimate statistic  p.value parameter conf.low conf.~1 method alter~2
##   <fct>       <dbl>     <dbl>    <dbl>     <int>    <dbl>   <dbl> <chr>  <chr>  
## 1 XI IPS 1    0.516     15.0  1.11e-43       622    0.456   0.571 Pears~ two.si~
## 2 XI IPS 2    0.332      8.44 2.54e-16       574    0.258   0.403 Pears~ two.si~
## 3 XI MIPA     0.341     10.0  2.65e-22       766    0.276   0.402 Pears~ two.si~
## # ... with abbreviated variable names 1: conf.high, 2: alternative
\end{verbatim}

\hypertarget{pca}{%
\subsection{PCA}\label{pca}}

\hypertarget{preprocess}{%
\subsubsection{Preprocess}\label{preprocess}}

\begin{Shaded}
\begin{Highlighting}[]
\NormalTok{data\_wide }\OtherTok{\textless{}{-}}
\NormalTok{  data }\SpecialCharTok{\%\textgreater{}\%}
  \FunctionTok{group\_by}\NormalTok{(name\_code) }\SpecialCharTok{\%\textgreater{}\%} 
  \FunctionTok{pivot\_wider}\NormalTok{(}
    \AttributeTok{names\_from =}\NormalTok{ ppp,}
    \AttributeTok{values\_from =}\NormalTok{ nilai\_ppp,}
    \AttributeTok{values\_fill =} \DecValTok{0}\NormalTok{,}
\NormalTok{  ) }\SpecialCharTok{\%\textgreater{}\%} 
  \FunctionTok{pivot\_wider}\NormalTok{(}
    \AttributeTok{names\_from =}\NormalTok{ penilaian\_h,}
    \AttributeTok{values\_from =}\NormalTok{ nilai\_ph,}
    \AttributeTok{values\_fill =} \DecValTok{0}
\NormalTok{  ) }\SpecialCharTok{\%\textgreater{}\%} 
  \FunctionTok{pivot\_wider}\NormalTok{(}
    \AttributeTok{names\_from =}\NormalTok{ penilaian\_s,}
    \AttributeTok{values\_from =}\NormalTok{ nilai,}
\NormalTok{  ) }\SpecialCharTok{\%\textgreater{}\%} 
  \FunctionTok{summarise\_all}\NormalTok{(mean) }\SpecialCharTok{\%\textgreater{}\%} 
  \FunctionTok{select}\NormalTok{(}\SpecialCharTok{{-}}\FunctionTok{c}\NormalTok{(kelas, materi, exp))}
  
\NormalTok{data\_wide}
\end{Highlighting}
\end{Shaded}

\begin{verbatim}
## # A tibble: 82 x 10
##    name_code  Proyek Praktek Porto~1 Haria~2 Haria~3 Haria~4 Haria~5   PAT  PHBO
##    <chr>       <dbl>   <dbl>   <dbl>   <dbl>   <dbl>   <dbl>   <dbl> <dbl> <dbl>
##  1 00eef26f8~   30      27.3    29.8    21.2    23.2    21      21.7    91    93
##  2 01b537880~   26      26      26      19.5    19.5    19.5    19.5    78    78
##  3 047c014f1~   26      26      26      19.5    19.5    19.5    19.5    85    88
##  4 05d012f70~   26.6    26      26      19.5    19.5    20.1    19.5    85    78
##  5 08d87310b~   26      26      26      19.5    19.5    19.5    19.5    85    78
##  6 0a3260cc8~   26      26      26      19.5    19.5    19.5    19.5    84    78
##  7 0d4fa2684~   26      26      26      19.5    19.5    19.5    19.5    86    78
##  8 0dc35a585~   27.8    27.8    27.3    21.2    19.5    20.8    21.2    88    90
##  9 110ad4f77~   27.3    27.2    26.9    21.2    19.5    20.8    19.8    89    94
## 10 120b3ac0b~   26.2    26.6    26.2    19.5    19.5    19.5    20.4    83    78
## # ... with 72 more rows, and abbreviated variable names 1: Portofolio,
## #   2: `Harian 4`, 3: `Harian 2`, 4: `Harian 3`, 5: `Harian 1`
## # i Use `print(n = ...)` to see more rows
\end{verbatim}

Use recipe to mark some column as ID, then normalise and process the
rest.

\begin{Shaded}
\begin{Highlighting}[]
\NormalTok{process\_rec }\OtherTok{\textless{}{-}} \CommentTok{\# define recipe}
  \FunctionTok{recipe}\NormalTok{(}\SpecialCharTok{\textasciitilde{}}\NormalTok{., }\AttributeTok{data =}\NormalTok{ data\_wide) }\SpecialCharTok{\%\textgreater{}\%} 
  \FunctionTok{update\_role}\NormalTok{(name\_code, }\AttributeTok{new\_role =} \StringTok{"id"}\NormalTok{) }\SpecialCharTok{\%\textgreater{}\%} 
  \FunctionTok{step\_dummy}\NormalTok{(}\FunctionTok{all\_nominal\_predictors}\NormalTok{()) }\SpecialCharTok{\%\textgreater{}\%} 
  \FunctionTok{step\_normalize}\NormalTok{(}\FunctionTok{all\_predictors}\NormalTok{()) }

\NormalTok{process\_prep }\OtherTok{\textless{}{-}} \FunctionTok{prep}\NormalTok{(process\_rec) }\CommentTok{\#evaluate recipe}
\end{Highlighting}
\end{Shaded}

Before applying pca, let's look at the prepared data.

\begin{Shaded}
\begin{Highlighting}[]
\FunctionTok{juice}\NormalTok{(process\_prep)}
\end{Highlighting}
\end{Shaded}

\begin{verbatim}
## # A tibble: 82 x 10
##    name_~1 Proyek Praktek Porto~2 Haria~3 Haria~4 Haria~5 Haria~6     PAT   PHBO
##    <fct>    <dbl>   <dbl>   <dbl>   <dbl>   <dbl>   <dbl>   <dbl>   <dbl>  <dbl>
##  1 00eef2~  3.35   1.20     3.40    1.55    3.57    1.55    2.16   1.92    1.22 
##  2 01b537~ -0.748 -0.887   -0.684  -0.639  -0.411  -0.715  -0.750 -1.83   -1.01 
##  3 047c01~ -0.748 -0.887   -0.684  -0.639  -0.411  -0.715  -0.750  0.190   0.477
##  4 05d012~ -0.150 -0.887   -0.684  -0.639  -0.411   0.163  -0.750  0.190  -1.01 
##  5 08d873~ -0.748 -0.887   -0.684  -0.639  -0.411  -0.715  -0.750  0.190  -1.01 
##  6 0a3260~ -0.748 -0.887   -0.684  -0.639  -0.411  -0.715  -0.750 -0.0983 -1.01 
##  7 0d4fa2~ -0.748 -0.887   -0.684  -0.639  -0.411  -0.715  -0.750  0.478  -1.01 
##  8 0dc35a~  1.05   1.85     0.768   1.55   -0.411   1.29    1.60   1.05    0.774
##  9 110ad4~  0.618  0.941    0.314   1.55   -0.411   1.29   -0.304  1.34    1.37 
## 10 120b3a~ -0.577  0.0271  -0.502  -0.639  -0.411  -0.715   0.481 -0.386  -1.01 
## # ... with 72 more rows, and abbreviated variable names 1: name_code,
## #   2: Portofolio, 3: `Harian 4`, 4: `Harian 2`, 5: `Harian 3`, 6: `Harian 1`
## # i Use `print(n = ...)` to see more rows
\end{verbatim}

Now let's apply step\_pca().

\begin{Shaded}
\begin{Highlighting}[]
\NormalTok{pca\_rec }\OtherTok{\textless{}{-}}
\NormalTok{  process\_rec }\SpecialCharTok{\%\textgreater{}\%} 
  \FunctionTok{step\_pca}\NormalTok{(}\FunctionTok{all\_predictors}\NormalTok{())}

\NormalTok{pca\_prep }\OtherTok{\textless{}{-}} \FunctionTok{prep}\NormalTok{(pca\_rec)}
\end{Highlighting}
\end{Shaded}

\begin{Shaded}
\begin{Highlighting}[]
\FunctionTok{names}\NormalTok{(pca\_prep)}
\end{Highlighting}
\end{Shaded}

\begin{verbatim}
##  [1] "var_info"       "term_info"      "steps"          "template"      
##  [5] "levels"         "retained"       "requirements"   "tr_info"       
##  [9] "orig_lvls"      "last_term_info"
\end{verbatim}

Look at the pca results.

\begin{Shaded}
\begin{Highlighting}[]
\FunctionTok{juice}\NormalTok{(pca\_prep)}
\end{Highlighting}
\end{Shaded}

\begin{verbatim}
## # A tibble: 82 x 6
##    name_code                                   PC1     PC2     PC3    PC4    PC5
##    <fct>                                     <dbl>   <dbl>   <dbl>  <dbl>  <dbl>
##  1 00eef26f828f1aaae411f94c8e475927548447fb~ -6.73  2.03   -0.0470 -0.795  1.21 
##  2 01b537880eb4733ff374caa583ac4fb7c1450beb~  2.46  0.644  -0.998   0.480 -0.272
##  3 047c014f151d6cef3519a4beb4e38d3b45ca024b~  1.51 -0.0756  1.03    0.367  0.463
##  4 05d012f707a50ab936f3d52b33582adccf7e015f~  1.39  0.0652  0.337  -0.705 -0.144
##  5 08d87310b7c81d621f0299cdcc647fb7a4f67a9f~  1.94  0.108   0.306  -0.767  0.207
##  6 0a3260cc81a1ebf7df96e3af1ac17223c6aad31f~  2.01  0.185   0.120  -0.589  0.139
##  7 0d4fa26847745631c4df3be6b4f3fc99f9893d8d~  1.86  0.0316  0.493  -0.946  0.276
##  8 0dc35a585cce664bed5839c0dd931c1e471a9915~ -3.19 -1.55   -0.284  -0.266 -0.691
##  9 110ad4f776bcbb0f46b89fbda4fc8ec8dd4dd0f0~ -2.10 -1.95    0.578   0.331  0.605
## 10 120b3ac0bd680cb3d22db096aa82d55510240f03~  1.19  0.257  -0.380  -0.611 -0.763
## # ... with 72 more rows
## # i Use `print(n = ...)` to see more rows
\end{verbatim}

\hypertarget{pc-contributions-to-variance-explained}{%
\paragraph{PC Contributions to Variance
Explained}\label{pc-contributions-to-variance-explained}}

Find standard deviation in each PC from steps.

\begin{Shaded}
\begin{Highlighting}[]
\NormalTok{sdev }\OtherTok{\textless{}{-}} 
\NormalTok{  pca\_prep}\SpecialCharTok{$}\NormalTok{steps[[}\DecValTok{3}\NormalTok{]]}\SpecialCharTok{$}\NormalTok{res}\SpecialCharTok{$}\NormalTok{sdev}
\end{Highlighting}
\end{Shaded}

Calculate percent of variation with standard deviation.

\begin{Shaded}
\begin{Highlighting}[]
\NormalTok{percent\_variation }\OtherTok{\textless{}{-}}\NormalTok{ sdev}\SpecialCharTok{\^{}}\DecValTok{2} \SpecialCharTok{/} \FunctionTok{sum}\NormalTok{(sdev}\SpecialCharTok{\^{}}\DecValTok{2}\NormalTok{)}
\end{Highlighting}
\end{Shaded}

\begin{Shaded}
\begin{Highlighting}[]
\NormalTok{var\_df }\OtherTok{\textless{}{-}} \FunctionTok{data.frame}\NormalTok{(}\AttributeTok{PC=}\FunctionTok{paste0}\NormalTok{(}\StringTok{"PC"}\NormalTok{,}\DecValTok{1}\SpecialCharTok{:}\FunctionTok{length}\NormalTok{(sdev)),}
                     \AttributeTok{var\_explained=}\NormalTok{percent\_variation,}
                     \AttributeTok{stringsAsFactors =} \ConstantTok{FALSE}\NormalTok{)}
\NormalTok{var\_df}
\end{Highlighting}
\end{Shaded}

\begin{verbatim}
##    PC var_explained
## 1 PC1  6.312045e-01
## 2 PC2  1.295237e-01
## 3 PC3  8.627531e-02
## 4 PC4  5.617098e-02
## 5 PC5  4.530960e-02
## 6 PC6  3.973402e-02
## 7 PC7  9.553585e-03
## 8 PC8  2.228303e-03
## 9 PC9  9.198070e-09
\end{verbatim}

Plot.

\begin{Shaded}
\begin{Highlighting}[]
\NormalTok{var\_df }\SpecialCharTok{\%\textgreater{}\%}
  \FunctionTok{mutate}\NormalTok{(}\AttributeTok{PC =} \FunctionTok{fct\_inorder}\NormalTok{(PC),}
         \AttributeTok{var\_rounded =} \FunctionTok{round}\NormalTok{(var\_explained, }\AttributeTok{digits =} \DecValTok{2}\NormalTok{),}
         \AttributeTok{var\_cum =} \FunctionTok{cumsum}\NormalTok{(var\_rounded)) }\SpecialCharTok{\%\textgreater{}\%}
  \FunctionTok{ggplot}\NormalTok{(}\FunctionTok{aes}\NormalTok{(}\AttributeTok{x =}\NormalTok{ PC,}
             \AttributeTok{y=}\NormalTok{ var\_explained, }
             \AttributeTok{fill =}\NormalTok{ PC))}\SpecialCharTok{+}
  \FunctionTok{geom\_col}\NormalTok{()}\SpecialCharTok{+} 
  \FunctionTok{scale\_fill\_manual}\NormalTok{(}
    \AttributeTok{values =} \FunctionTok{c}\NormalTok{(}
      \StringTok{"\#C6E0FF"}\NormalTok{,}
      \StringTok{"\#579A9E"}\NormalTok{,}
      \StringTok{"\#3292C3"}\NormalTok{,}
      \StringTok{"\#BCAB79"}\NormalTok{,}
      \StringTok{"maroon"}\NormalTok{,}
      \StringTok{"white"}\NormalTok{,}
      \StringTok{"violet"}\NormalTok{,}
      \StringTok{"grey"}\NormalTok{,}
      \StringTok{"brown"}\NormalTok{,}
      \StringTok{"lightyellow"}
\NormalTok{    )}
\NormalTok{  )}\SpecialCharTok{+}
  \FunctionTok{geom\_label}\NormalTok{(}\FunctionTok{aes}\NormalTok{(}\AttributeTok{label =}\NormalTok{ var\_cum))}\SpecialCharTok{+}
  \FunctionTok{labs}\NormalTok{(}\AttributeTok{title =} \StringTok{"PC Contributions to Variance Explained"}\NormalTok{,}
       \AttributeTok{x =} \ConstantTok{NULL}\NormalTok{,}
       \AttributeTok{y =} \StringTok{"Var Explained"}\NormalTok{)}\SpecialCharTok{+}
  \FunctionTok{theme}\NormalTok{(}\AttributeTok{legend.position =} \StringTok{"none"}\NormalTok{)}
\end{Highlighting}
\end{Shaded}

\includegraphics{unsupervised-learning_files/figure-latex/unnamed-chunk-63-1.pdf}

It takes 4 PC to retain 91\% variance explained.

\begin{Shaded}
\begin{Highlighting}[]
\NormalTok{tidied\_pca }\OtherTok{\textless{}{-}} \FunctionTok{tidy}\NormalTok{(pca\_prep, }
                   \AttributeTok{number =} \DecValTok{3}\NormalTok{) }\CommentTok{\# the number of step in recipe, normalize is 2, pca is 3}
\end{Highlighting}
\end{Shaded}

\hypertarget{visualize-principal-components}{%
\paragraph{Visualize Principal
Components}\label{visualize-principal-components}}

\begin{Shaded}
\begin{Highlighting}[]
\NormalTok{tidied\_pca }\SpecialCharTok{\%\textgreater{}\%}
  \FunctionTok{filter}\NormalTok{(component }\SpecialCharTok{\%in\%} \FunctionTok{paste0}\NormalTok{(}\StringTok{"PC"}\NormalTok{, }\DecValTok{1}\SpecialCharTok{:}\DecValTok{6}\NormalTok{)) }\SpecialCharTok{\%\textgreater{}\%}
  \FunctionTok{mutate}\NormalTok{(}\AttributeTok{component =} \FunctionTok{fct\_inorder}\NormalTok{(component)) }\SpecialCharTok{\%\textgreater{}\%}
  \FunctionTok{ggplot}\NormalTok{(}\FunctionTok{aes}\NormalTok{(value, terms, }\AttributeTok{fill =}\NormalTok{ terms)) }\SpecialCharTok{+}
  \FunctionTok{geom\_col}\NormalTok{(}\AttributeTok{show.legend =} \ConstantTok{FALSE}\NormalTok{) }\SpecialCharTok{+}
  \FunctionTok{facet\_wrap}\NormalTok{(}\SpecialCharTok{\textasciitilde{}}\NormalTok{component, }\AttributeTok{nrow =} \DecValTok{1}\NormalTok{) }\SpecialCharTok{+}
  \FunctionTok{labs}\NormalTok{(}\AttributeTok{y =} \ConstantTok{NULL}\NormalTok{)}\SpecialCharTok{+}
  \FunctionTok{scale\_fill\_manual}\NormalTok{(}
    \AttributeTok{values =} \FunctionTok{c}\NormalTok{(}
      \StringTok{"maroon"}\NormalTok{,}
      \StringTok{"white"}\NormalTok{,}
      \StringTok{"violet"}\NormalTok{,}
      \StringTok{"grey"}\NormalTok{,}
      \StringTok{"brown"}\NormalTok{,}
      \StringTok{"lightyellow"}\NormalTok{,}
      \StringTok{"\#C6E0FF"}\NormalTok{,}
      \StringTok{"\#579A9E"}\NormalTok{,}
      \StringTok{"\#3292C3"}\NormalTok{,}
      \StringTok{"\#BCAB79"}
\NormalTok{    )}
\NormalTok{  )}
\end{Highlighting}
\end{Shaded}

\includegraphics{unsupervised-learning_files/figure-latex/unnamed-chunk-65-1.pdf}

Focus on the first 4 PC.

\begin{Shaded}
\begin{Highlighting}[]
\NormalTok{tidied\_pca }\SpecialCharTok{\%\textgreater{}\%}
  \FunctionTok{filter}\NormalTok{(component }\SpecialCharTok{\%in\%} \FunctionTok{paste0}\NormalTok{(}\StringTok{"PC"}\NormalTok{, }\DecValTok{1}\SpecialCharTok{:}\DecValTok{4}\NormalTok{)) }\SpecialCharTok{\%\textgreater{}\%}
  \FunctionTok{group\_by}\NormalTok{(component) }\SpecialCharTok{\%\textgreater{}\%}
  \FunctionTok{top\_n}\NormalTok{(}\DecValTok{8}\NormalTok{, }\FunctionTok{abs}\NormalTok{(value)) }\SpecialCharTok{\%\textgreater{}\%}
  \FunctionTok{ungroup}\NormalTok{() }\SpecialCharTok{\%\textgreater{}\%}
  \FunctionTok{mutate}\NormalTok{(}\AttributeTok{terms =} \FunctionTok{reorder\_within}\NormalTok{(terms, }\FunctionTok{abs}\NormalTok{(value), component)) }\SpecialCharTok{\%\textgreater{}\%}
  \FunctionTok{ggplot}\NormalTok{(}\FunctionTok{aes}\NormalTok{(}\FunctionTok{abs}\NormalTok{(value), terms, }\AttributeTok{fill =}\NormalTok{ value }\SpecialCharTok{\textgreater{}} \DecValTok{0}\NormalTok{)) }\SpecialCharTok{+}
  \FunctionTok{geom\_col}\NormalTok{() }\SpecialCharTok{+}
  \FunctionTok{facet\_wrap}\NormalTok{(}\SpecialCharTok{\textasciitilde{}}\NormalTok{component, }\AttributeTok{scales =} \StringTok{"free\_y"}\NormalTok{) }\SpecialCharTok{+}
  \FunctionTok{scale\_y\_reordered}\NormalTok{() }\SpecialCharTok{+}
  \FunctionTok{labs}\NormalTok{(}
    \AttributeTok{x =} \StringTok{"Absolute value of contribution"}\NormalTok{,}
    \AttributeTok{y =} \ConstantTok{NULL}\NormalTok{, }\AttributeTok{fill =} \StringTok{"Positive?"}
\NormalTok{  )}\SpecialCharTok{+}
  \FunctionTok{scale\_fill\_manual}\NormalTok{(}
    \AttributeTok{values =} \FunctionTok{c}\NormalTok{(}\StringTok{"\#BEEF9E"}\NormalTok{, }\StringTok{"\#3292C3"}\NormalTok{)}
\NormalTok{  )}
\end{Highlighting}
\end{Shaded}

\includegraphics{unsupervised-learning_files/figure-latex/unnamed-chunk-66-1.pdf}

The biggest divergence is at PC2, between Harian 2 and Harian 4.

\begin{Shaded}
\begin{Highlighting}[]
\CommentTok{\# use juice() to transform int dataframe}
\NormalTok{juiced\_pca }\OtherTok{\textless{}{-}} \FunctionTok{juice}\NormalTok{(pca\_prep)}
\NormalTok{juiced\_pca }
\end{Highlighting}
\end{Shaded}

\begin{verbatim}
## # A tibble: 82 x 6
##    name_code                                   PC1     PC2     PC3    PC4    PC5
##    <fct>                                     <dbl>   <dbl>   <dbl>  <dbl>  <dbl>
##  1 00eef26f828f1aaae411f94c8e475927548447fb~ -6.73  2.03   -0.0470 -0.795  1.21 
##  2 01b537880eb4733ff374caa583ac4fb7c1450beb~  2.46  0.644  -0.998   0.480 -0.272
##  3 047c014f151d6cef3519a4beb4e38d3b45ca024b~  1.51 -0.0756  1.03    0.367  0.463
##  4 05d012f707a50ab936f3d52b33582adccf7e015f~  1.39  0.0652  0.337  -0.705 -0.144
##  5 08d87310b7c81d621f0299cdcc647fb7a4f67a9f~  1.94  0.108   0.306  -0.767  0.207
##  6 0a3260cc81a1ebf7df96e3af1ac17223c6aad31f~  2.01  0.185   0.120  -0.589  0.139
##  7 0d4fa26847745631c4df3be6b4f3fc99f9893d8d~  1.86  0.0316  0.493  -0.946  0.276
##  8 0dc35a585cce664bed5839c0dd931c1e471a9915~ -3.19 -1.55   -0.284  -0.266 -0.691
##  9 110ad4f776bcbb0f46b89fbda4fc8ec8dd4dd0f0~ -2.10 -1.95    0.578   0.331  0.605
## 10 120b3ac0bd680cb3d22db096aa82d55510240f03~  1.19  0.257  -0.380  -0.611 -0.763
## # ... with 72 more rows
## # i Use `print(n = ...)` to see more rows
\end{verbatim}

\begin{Shaded}
\begin{Highlighting}[]
  \FunctionTok{ggplot}\NormalTok{(juiced\_pca, }\FunctionTok{aes}\NormalTok{(PC1, PC2)) }\SpecialCharTok{+}
  \FunctionTok{geom\_point}\NormalTok{(}\AttributeTok{alpha =} \FloatTok{0.7}\NormalTok{, }\AttributeTok{size =} \DecValTok{2}\NormalTok{) }\SpecialCharTok{+}
  \FunctionTok{labs}\NormalTok{(}\AttributeTok{color =} \ConstantTok{NULL}\NormalTok{)}\SpecialCharTok{+}
  \FunctionTok{theme\_minimal}\NormalTok{()}
\end{Highlighting}
\end{Shaded}

\includegraphics{unsupervised-learning_files/figure-latex/unnamed-chunk-68-1.pdf}

\begin{Shaded}
\begin{Highlighting}[]
  \FunctionTok{ggplot}\NormalTok{(juiced\_pca, }\FunctionTok{aes}\NormalTok{(PC1, PC3)) }\SpecialCharTok{+}
  \FunctionTok{geom\_point}\NormalTok{(}\AttributeTok{alpha =} \FloatTok{0.7}\NormalTok{, }\AttributeTok{size =} \DecValTok{2}\NormalTok{) }\SpecialCharTok{+}
  \FunctionTok{labs}\NormalTok{(}\AttributeTok{color =} \ConstantTok{NULL}\NormalTok{)}\SpecialCharTok{+}
  \FunctionTok{theme\_minimal}\NormalTok{()}
\end{Highlighting}
\end{Shaded}

\includegraphics{unsupervised-learning_files/figure-latex/unnamed-chunk-68-2.pdf}

\begin{Shaded}
\begin{Highlighting}[]
  \FunctionTok{ggplot}\NormalTok{(juiced\_pca, }\FunctionTok{aes}\NormalTok{(PC1, PC4)) }\SpecialCharTok{+}
  \FunctionTok{geom\_point}\NormalTok{(}\AttributeTok{alpha =} \FloatTok{0.7}\NormalTok{, }\AttributeTok{size =} \DecValTok{2}\NormalTok{) }\SpecialCharTok{+}
  \FunctionTok{labs}\NormalTok{(}\AttributeTok{color =} \ConstantTok{NULL}\NormalTok{)}\SpecialCharTok{+}
  \FunctionTok{theme\_minimal}\NormalTok{()}
\end{Highlighting}
\end{Shaded}

\includegraphics{unsupervised-learning_files/figure-latex/unnamed-chunk-68-3.pdf}

\hypertarget{pca-with-factominer}{%
\paragraph{PCA with FactoMiner}\label{pca-with-factominer}}

\begin{Shaded}
\begin{Highlighting}[]
\NormalTok{pca\_rec2 }\OtherTok{\textless{}{-}} \CommentTok{\# define recipe}
  \FunctionTok{recipe}\NormalTok{(}\SpecialCharTok{\textasciitilde{}}\NormalTok{., }\AttributeTok{data =}\NormalTok{ data\_wide) }\SpecialCharTok{\%\textgreater{}\%} 
  \FunctionTok{update\_role}\NormalTok{(name\_code, }\AttributeTok{new\_role =} \StringTok{"id"}\NormalTok{) }\SpecialCharTok{\%\textgreater{}\%} 
  \FunctionTok{step\_normalize}\NormalTok{(}\FunctionTok{all\_predictors}\NormalTok{()) }\SpecialCharTok{\%\textgreater{}\%} 
  \FunctionTok{prep}\NormalTok{()}
\end{Highlighting}
\end{Shaded}

\begin{Shaded}
\begin{Highlighting}[]
\NormalTok{d4facto }\OtherTok{\textless{}{-}} \FunctionTok{juice}\NormalTok{(pca\_rec2)}
\end{Highlighting}
\end{Shaded}

\begin{Shaded}
\begin{Highlighting}[]
\NormalTok{d4facto}
\end{Highlighting}
\end{Shaded}

\begin{verbatim}
## # A tibble: 82 x 10
##    name_~1 Proyek Praktek Porto~2 Haria~3 Haria~4 Haria~5 Haria~6     PAT   PHBO
##    <fct>    <dbl>   <dbl>   <dbl>   <dbl>   <dbl>   <dbl>   <dbl>   <dbl>  <dbl>
##  1 00eef2~  3.35   1.20     3.40    1.55    3.57    1.55    2.16   1.92    1.22 
##  2 01b537~ -0.748 -0.887   -0.684  -0.639  -0.411  -0.715  -0.750 -1.83   -1.01 
##  3 047c01~ -0.748 -0.887   -0.684  -0.639  -0.411  -0.715  -0.750  0.190   0.477
##  4 05d012~ -0.150 -0.887   -0.684  -0.639  -0.411   0.163  -0.750  0.190  -1.01 
##  5 08d873~ -0.748 -0.887   -0.684  -0.639  -0.411  -0.715  -0.750  0.190  -1.01 
##  6 0a3260~ -0.748 -0.887   -0.684  -0.639  -0.411  -0.715  -0.750 -0.0983 -1.01 
##  7 0d4fa2~ -0.748 -0.887   -0.684  -0.639  -0.411  -0.715  -0.750  0.478  -1.01 
##  8 0dc35a~  1.05   1.85     0.768   1.55   -0.411   1.29    1.60   1.05    0.774
##  9 110ad4~  0.618  0.941    0.314   1.55   -0.411   1.29   -0.304  1.34    1.37 
## 10 120b3a~ -0.577  0.0271  -0.502  -0.639  -0.411  -0.715   0.481 -0.386  -1.01 
## # ... with 72 more rows, and abbreviated variable names 1: name_code,
## #   2: Portofolio, 3: `Harian 4`, 4: `Harian 2`, 5: `Harian 3`, 6: `Harian 1`
## # i Use `print(n = ...)` to see more rows
\end{verbatim}

\begin{Shaded}
\begin{Highlighting}[]
\CommentTok{\# library(FactoMineR)}

\NormalTok{facto\_pca }\OtherTok{\textless{}{-}} \FunctionTok{PCA}\NormalTok{(}
  \AttributeTok{X =}\NormalTok{ d4facto,}
  \AttributeTok{scale.unit =}\NormalTok{ F, }
  \AttributeTok{quali.sup =} \FunctionTok{c}\NormalTok{(}\DecValTok{1}\NormalTok{), }
  \AttributeTok{ncp =} \DecValTok{6}\NormalTok{, }
  \AttributeTok{graph =}\NormalTok{ F }\CommentTok{\#disable visualization}
\NormalTok{  )}
\end{Highlighting}
\end{Shaded}

\begin{Shaded}
\begin{Highlighting}[]
\FunctionTok{head}\NormalTok{(facto\_pca}\SpecialCharTok{$}\NormalTok{ind}\SpecialCharTok{$}\NormalTok{coord)}
\end{Highlighting}
\end{Shaded}

\begin{verbatim}
##       Dim.1       Dim.2       Dim.3      Dim.4      Dim.5      Dim.6
## 1  6.730412 -2.03072138 -0.04698073  0.7946486  1.2067587 -0.1802793
## 2 -2.455811 -0.64446443 -0.99755998 -0.4798995 -0.2719070 -0.2145915
## 3 -1.508593  0.07555772  1.03099615 -0.3669759  0.4632670  0.1705318
## 4 -1.388104 -0.06519872  0.33714264  0.7051051 -0.1443258 -0.8125022
## 5 -1.937431 -0.10820413  0.30631482  0.7674604  0.2073614 -0.1603130
## 6 -2.011485 -0.18481274  0.12004699  0.5892661  0.1388945 -0.1680670
\end{verbatim}

\begin{Shaded}
\begin{Highlighting}[]
\FunctionTok{plot.PCA}\NormalTok{(}
  \AttributeTok{x =}\NormalTok{ facto\_pca, }
  \AttributeTok{choix =} \StringTok{"ind"}\NormalTok{, }\CommentTok{\# individual factor map\# coloring}
  \AttributeTok{select =} \StringTok{"contrib 1"}\NormalTok{, }\CommentTok{\# 5 most contributing factor}
  \AttributeTok{invisible =} \StringTok{"quali"}
\NormalTok{)}
\end{Highlighting}
\end{Shaded}

\includegraphics{unsupervised-learning_files/figure-latex/unnamed-chunk-74-1.pdf}

\begin{Shaded}
\begin{Highlighting}[]
\NormalTok{d4facto[}\DecValTok{71}\NormalTok{,]}
\end{Highlighting}
\end{Shaded}

\begin{verbatim}
## # A tibble: 1 x 10
##   name_code   Proyek Praktek Porto~1 Haria~2 Haria~3 Haria~4 Haria~5   PAT  PHBO
##   <fct>        <dbl>   <dbl>   <dbl>   <dbl>   <dbl>   <dbl>   <dbl> <dbl> <dbl>
## 1 994d59d187~ -0.748   0.680  -0.684  -0.639  -0.411  -0.715   0.592 0.190 -1.01
## # ... with abbreviated variable names 1: Portofolio, 2: `Harian 4`,
## #   3: `Harian 2`, 4: `Harian 3`, 5: `Harian 1`
\end{verbatim}

\begin{Shaded}
\begin{Highlighting}[]
\NormalTok{thename }\OtherTok{\textless{}{-}}\NormalTok{ data\_wide[}\DecValTok{71}\NormalTok{,] }\SpecialCharTok{\%\textgreater{}\%} \FunctionTok{pull}\NormalTok{(name\_code)}
\end{Highlighting}
\end{Shaded}

\begin{Shaded}
\begin{Highlighting}[]
\NormalTok{data\_wide }\SpecialCharTok{\%\textgreater{}\%} 
  \FunctionTok{filter}\NormalTok{(name\_code }\SpecialCharTok{==}\NormalTok{ thename)}
\end{Highlighting}
\end{Shaded}

\begin{verbatim}
## # A tibble: 1 x 10
##   name_code   Proyek Praktek Porto~1 Haria~2 Haria~3 Haria~4 Haria~5   PAT  PHBO
##   <chr>        <dbl>   <dbl>   <dbl>   <dbl>   <dbl>   <dbl>   <dbl> <dbl> <dbl>
## 1 994d59d187~     26      27      26    19.5    19.5    19.5    20.5    85    78
## # ... with abbreviated variable names 1: Portofolio, 2: `Harian 4`,
## #   3: `Harian 2`, 4: `Harian 3`, 5: `Harian 1`
\end{verbatim}

\begin{Shaded}
\begin{Highlighting}[]
\FunctionTok{plot.PCA}\NormalTok{(}
  \AttributeTok{x =}\NormalTok{ facto\_pca, }
  \AttributeTok{choix =} \StringTok{"var"}\NormalTok{,}
  \AttributeTok{autoLab =} \StringTok{"yes"}
\NormalTok{)}
\end{Highlighting}
\end{Shaded}

\includegraphics{unsupervised-learning_files/figure-latex/unnamed-chunk-78-1.pdf}

\begin{Shaded}
\begin{Highlighting}[]
\FunctionTok{plot.PCA}\NormalTok{(}
  \AttributeTok{x =}\NormalTok{ facto\_pca, }
  \AttributeTok{choix =} \StringTok{"var"}\NormalTok{,}
  \AttributeTok{select =} \StringTok{"contrib 1"}
\NormalTok{)}
\end{Highlighting}
\end{Shaded}

\includegraphics{unsupervised-learning_files/figure-latex/unnamed-chunk-79-1.pdf}

The ``Ranah Keterampilan'' that's most correlated with the semester exam
score (PAT) is ``Praktek'' assignment. But, the variable that has the
most contribution the the two dimension the graph was built on, is
``Proyek''.

\hypertarget{k-means-clustering}{%
\subsection{K-means Clustering}\label{k-means-clustering}}

Reuse d4facto, but without the first 3 columns.

\begin{Shaded}
\begin{Highlighting}[]
\NormalTok{d4kmeans }\OtherTok{\textless{}{-}}\NormalTok{ d4facto }\SpecialCharTok{\%\textgreater{}\%} 
  \FunctionTok{select}\NormalTok{(}\SpecialCharTok{{-}}\DecValTok{1}\NormalTok{)}
\FunctionTok{head}\NormalTok{(d4kmeans)}
\end{Highlighting}
\end{Shaded}

\begin{verbatim}
## # A tibble: 6 x 9
##   Proyek Praktek Portofolio `Harian 4` `Harian 2` Haria~1 Haria~2     PAT   PHBO
##    <dbl>   <dbl>      <dbl>      <dbl>      <dbl>   <dbl>   <dbl>   <dbl>  <dbl>
## 1  3.35    1.20       3.40       1.55       3.57    1.55    2.16   1.92    1.22 
## 2 -0.748  -0.887     -0.684     -0.639     -0.411  -0.715  -0.750 -1.83   -1.01 
## 3 -0.748  -0.887     -0.684     -0.639     -0.411  -0.715  -0.750  0.190   0.477
## 4 -0.150  -0.887     -0.684     -0.639     -0.411   0.163  -0.750  0.190  -1.01 
## 5 -0.748  -0.887     -0.684     -0.639     -0.411  -0.715  -0.750  0.190  -1.01 
## 6 -0.748  -0.887     -0.684     -0.639     -0.411  -0.715  -0.750 -0.0983 -1.01 
## # ... with abbreviated variable names 1: `Harian 3`, 2: `Harian 1`
\end{verbatim}

\hypertarget{find-k}{%
\subsubsection{Find K}\label{find-k}}

We have to decide the numbers of K. Let's simulate 9 Ks.

\begin{Shaded}
\begin{Highlighting}[]
\NormalTok{kmsim }\OtherTok{\textless{}{-}} 
\FunctionTok{tibble}\NormalTok{(}\AttributeTok{k =} \DecValTok{1}\SpecialCharTok{:}\DecValTok{9}\NormalTok{)}

\NormalTok{kmsim}
\end{Highlighting}
\end{Shaded}

\begin{verbatim}
## # A tibble: 9 x 1
##       k
##   <int>
## 1     1
## 2     2
## 3     3
## 4     4
## 5     5
## 6     6
## 7     7
## 8     8
## 9     9
\end{verbatim}

Create kmeans clustering for each number of K.

\begin{Shaded}
\begin{Highlighting}[]
\FunctionTok{set.seed}\NormalTok{(}\DecValTok{1}\NormalTok{)}
\NormalTok{kmsim }\OtherTok{\textless{}{-}}\NormalTok{ kmsim }\SpecialCharTok{\%\textgreater{}\%} 
\FunctionTok{mutate}\NormalTok{(}
    \CommentTok{\# cluster the data 9 times}
    \AttributeTok{kclust =} \FunctionTok{map}\NormalTok{(k, }\SpecialCharTok{\textasciitilde{}} \FunctionTok{kmeans}\NormalTok{(d4kmeans, .))}
\NormalTok{  )}

\NormalTok{kmsim }
\end{Highlighting}
\end{Shaded}

\begin{verbatim}
## # A tibble: 9 x 2
##       k kclust  
##   <int> <list>  
## 1     1 <kmeans>
## 2     2 <kmeans>
## 3     3 <kmeans>
## 4     4 <kmeans>
## 5     5 <kmeans>
## 6     6 <kmeans>
## 7     7 <kmeans>
## 8     8 <kmeans>
## 9     9 <kmeans>
\end{verbatim}

\hypertarget{glance}{%
\paragraph{Glance}\label{glance}}

Apply tidy, glance, and augment to each kclust.

\begin{Shaded}
\begin{Highlighting}[]
\FunctionTok{set.seed}\NormalTok{(}\DecValTok{1}\NormalTok{)}
\NormalTok{kmsim }\OtherTok{\textless{}{-}}\NormalTok{ kmsim }\SpecialCharTok{\%\textgreater{}\%} 
\FunctionTok{mutate}\NormalTok{(}
  \AttributeTok{glanced =} \FunctionTok{map}\NormalTok{(kclust, glance)}
\NormalTok{)}
\NormalTok{kmsim}
\end{Highlighting}
\end{Shaded}

\begin{verbatim}
## # A tibble: 9 x 3
##       k kclust   glanced         
##   <int> <list>   <list>          
## 1     1 <kmeans> <tibble [1 x 4]>
## 2     2 <kmeans> <tibble [1 x 4]>
## 3     3 <kmeans> <tibble [1 x 4]>
## 4     4 <kmeans> <tibble [1 x 4]>
## 5     5 <kmeans> <tibble [1 x 4]>
## 6     6 <kmeans> <tibble [1 x 4]>
## 7     7 <kmeans> <tibble [1 x 4]>
## 8     8 <kmeans> <tibble [1 x 4]>
## 9     9 <kmeans> <tibble [1 x 4]>
\end{verbatim}

Unnest glanced

\begin{Shaded}
\begin{Highlighting}[]
\NormalTok{clustering }\OtherTok{\textless{}{-}}
\NormalTok{  kmsim }\SpecialCharTok{\%\textgreater{}\%} 
  \FunctionTok{unnest}\NormalTok{(}\AttributeTok{cols =}\NormalTok{ glanced)}
\NormalTok{clustering }\SpecialCharTok{\%\textgreater{}\%} \FunctionTok{head}\NormalTok{()}
\end{Highlighting}
\end{Shaded}

\begin{verbatim}
## # A tibble: 6 x 6
##       k kclust   totss tot.withinss betweenss  iter
##   <int> <list>   <dbl>        <dbl>     <dbl> <int>
## 1     1 <kmeans>   729         729   1.14e-13     1
## 2     2 <kmeans>   729         390.  3.39e+ 2     1
## 3     3 <kmeans>   729         288.  4.41e+ 2     2
## 4     4 <kmeans>   729         253.  4.76e+ 2     3
## 5     5 <kmeans>   729         217.  5.12e+ 2     3
## 6     6 <kmeans>   729         193.  5.36e+ 2     3
\end{verbatim}

\begin{Shaded}
\begin{Highlighting}[]
\FunctionTok{ggplot}\NormalTok{(clustering, }\FunctionTok{aes}\NormalTok{(k, tot.withinss))}\SpecialCharTok{+}
  \FunctionTok{geom\_line}\NormalTok{()}\SpecialCharTok{+}
  \FunctionTok{geom\_point}\NormalTok{()}
\end{Highlighting}
\end{Shaded}

\includegraphics{unsupervised-learning_files/figure-latex/unnamed-chunk-85-1.pdf}

There are 3 centers before the tot.withinss decrease slowed down. That's
a good enough elbow. So our kmeans is:

\begin{Shaded}
\begin{Highlighting}[]
\NormalTok{thek }\OtherTok{\textless{}{-}}\NormalTok{ kmsim }\SpecialCharTok{\%\textgreater{}\%} 
  \FunctionTok{filter}\NormalTok{(k }\SpecialCharTok{==} \DecValTok{3}\NormalTok{) }\SpecialCharTok{\%\textgreater{}\%} 
  \FunctionTok{pull}\NormalTok{(kclust)}
\end{Highlighting}
\end{Shaded}

Assign the cluster identification to a new column in d4kmeans

\begin{Shaded}
\begin{Highlighting}[]
\NormalTok{d4kmeans}\SpecialCharTok{$}\NormalTok{cluster }\OtherTok{\textless{}{-}}\NormalTok{ thek[[}\DecValTok{1}\NormalTok{]][[}\DecValTok{1}\NormalTok{]]}
\end{Highlighting}
\end{Shaded}

\hypertarget{profiling}{%
\subsubsection{Profiling}\label{profiling}}

\begin{Shaded}
\begin{Highlighting}[]
\NormalTok{student\_score\_profile }\OtherTok{\textless{}{-}} 
\NormalTok{d4kmeans }\SpecialCharTok{\%\textgreater{}\%} 
  \FunctionTok{group\_by}\NormalTok{(cluster) }\SpecialCharTok{\%\textgreater{}\%} 
  \FunctionTok{summarise\_all}\NormalTok{(mean)}
\NormalTok{student\_score\_profile}
\end{Highlighting}
\end{Shaded}

\begin{verbatim}
## # A tibble: 3 x 10
##   cluster Proyek Praktek Portofo~1 Haria~2 Haria~3 Haria~4 Haria~5    PAT   PHBO
##     <int>  <dbl>   <dbl>     <dbl>   <dbl>   <dbl>   <dbl>   <dbl>  <dbl>  <dbl>
## 1       1 -0.677  -0.724    -0.630  -0.639  -0.411  -0.665  -0.496 -0.457 -0.587
## 2       2  1.88    0.745     1.98    0.271   2.21    1.07    1.56   0.646  0.935
## 3       3  0.315   0.946     0.184   1.02   -0.324   0.681   0.144  0.512  0.607
## # ... with abbreviated variable names 1: Portofolio, 2: `Harian 4`,
## #   3: `Harian 2`, 4: `Harian 3`, 5: `Harian 1`
\end{verbatim}

\begin{Shaded}
\begin{Highlighting}[]
\FunctionTok{library}\NormalTok{(ggiraphExtra)}
\FunctionTok{ggRadar}\NormalTok{(}
  \AttributeTok{data=}\NormalTok{student\_score\_profile,}
  \AttributeTok{mapping =} \FunctionTok{aes}\NormalTok{(}\AttributeTok{colours =}\NormalTok{ cluster),}
  \AttributeTok{ylim =} \FloatTok{0.8}
\NormalTok{)}\SpecialCharTok{+}
  \FunctionTok{facet\_wrap}\NormalTok{(}\SpecialCharTok{\textasciitilde{}}\NormalTok{cluster)}
\end{Highlighting}
\end{Shaded}

\includegraphics{unsupervised-learning_files/figure-latex/unnamed-chunk-89-1.pdf}

\begin{Shaded}
\begin{Highlighting}[]
\NormalTok{d4kmeans }\SpecialCharTok{\%\textgreater{}\%} 
  \FunctionTok{head}\NormalTok{()}
\end{Highlighting}
\end{Shaded}

\begin{verbatim}
## # A tibble: 6 x 10
##   Proyek Praktek Portof~1 Haria~2 Haria~3 Haria~4 Haria~5     PAT   PHBO cluster
##    <dbl>   <dbl>    <dbl>   <dbl>   <dbl>   <dbl>   <dbl>   <dbl>  <dbl>   <int>
## 1  3.35    1.20     3.40    1.55    3.57    1.55    2.16   1.92    1.22        2
## 2 -0.748  -0.887   -0.684  -0.639  -0.411  -0.715  -0.750 -1.83   -1.01        1
## 3 -0.748  -0.887   -0.684  -0.639  -0.411  -0.715  -0.750  0.190   0.477       1
## 4 -0.150  -0.887   -0.684  -0.639  -0.411   0.163  -0.750  0.190  -1.01        1
## 5 -0.748  -0.887   -0.684  -0.639  -0.411  -0.715  -0.750  0.190  -1.01        1
## 6 -0.748  -0.887   -0.684  -0.639  -0.411  -0.715  -0.750 -0.0983 -1.01        1
## # ... with abbreviated variable names 1: Portofolio, 2: `Harian 4`,
## #   3: `Harian 2`, 4: `Harian 3`, 5: `Harian 1`
\end{verbatim}

\hypertarget{viz}{%
\paragraph{Viz}\label{viz}}

\begin{Shaded}
\begin{Highlighting}[]
\CommentTok{\#library(factoextra)}
\FunctionTok{set.seed}\NormalTok{(}\DecValTok{1}\NormalTok{)}
\NormalTok{x }\OtherTok{\textless{}{-}} \FunctionTok{kmeans}\NormalTok{(d4kmeans, }\DecValTok{3}\NormalTok{)}

\FunctionTok{fviz\_cluster}\NormalTok{(}\AttributeTok{object =}\NormalTok{ x, }
             \AttributeTok{data =}\NormalTok{ d4kmeans)}
\end{Highlighting}
\end{Shaded}

\includegraphics{unsupervised-learning_files/figure-latex/unnamed-chunk-91-1.pdf}

\hypertarget{combine}{%
\subsection{Combine}\label{combine}}

\begin{Shaded}
\begin{Highlighting}[]
\NormalTok{km\_x\_pca }\OtherTok{\textless{}{-}} \FunctionTok{PCA}\NormalTok{(}\AttributeTok{X =}\NormalTok{ d4kmeans, }\CommentTok{\# this table has already added with the cluster}
               \AttributeTok{quali.sup =} \DecValTok{10}\NormalTok{, }
               \AttributeTok{graph=}\NormalTok{F) }

\FunctionTok{fviz\_pca\_biplot}\NormalTok{(km\_x\_pca,}
                \AttributeTok{habillage =} \StringTok{"cluster"}\NormalTok{,}
                \AttributeTok{geom.ind =} \StringTok{"point"}\NormalTok{, }\CommentTok{\# menampilkan titik observasi saja}
                \AttributeTok{addEllipses =}\NormalTok{ T, }\CommentTok{\# membuat elips disekitar cluster}
                \AttributeTok{col.var =} \StringTok{"navy"}
\NormalTok{)}
\end{Highlighting}
\end{Shaded}

\includegraphics{unsupervised-learning_files/figure-latex/unnamed-chunk-92-1.pdf}

Summary:

\begin{itemize}
\tightlist
\item
\end{itemize}

\end{document}
